\documentclass[]{book}
\usepackage{lmodern}
\usepackage{amssymb,amsmath}
\usepackage{ifxetex,ifluatex}
\usepackage{fixltx2e} % provides \textsubscript
\ifnum 0\ifxetex 1\fi\ifluatex 1\fi=0 % if pdftex
  \usepackage[T1]{fontenc}
  \usepackage[utf8]{inputenc}
\else % if luatex or xelatex
  \ifxetex
    \usepackage{mathspec}
  \else
    \usepackage{fontspec}
  \fi
  \defaultfontfeatures{Ligatures=TeX,Scale=MatchLowercase}
\fi
% use upquote if available, for straight quotes in verbatim environments
\IfFileExists{upquote.sty}{\usepackage{upquote}}{}
% use microtype if available
\IfFileExists{microtype.sty}{%
\usepackage{microtype}
\UseMicrotypeSet[protrusion]{basicmath} % disable protrusion for tt fonts
}{}
\usepackage[margin=1in]{geometry}
\usepackage{hyperref}
\hypersetup{unicode=true,
            pdftitle={Data Science With Python},
            pdfauthor={L A Liggett},
            pdfborder={0 0 0},
            breaklinks=true}
\urlstyle{same}  % don't use monospace font for urls
\usepackage{natbib}
\bibliographystyle{apalike}
\usepackage{color}
\usepackage{fancyvrb}
\newcommand{\VerbBar}{|}
\newcommand{\VERB}{\Verb[commandchars=\\\{\}]}
\DefineVerbatimEnvironment{Highlighting}{Verbatim}{commandchars=\\\{\}}
% Add ',fontsize=\small' for more characters per line
\usepackage{framed}
\definecolor{shadecolor}{RGB}{248,248,248}
\newenvironment{Shaded}{\begin{snugshade}}{\end{snugshade}}
\newcommand{\KeywordTok}[1]{\textcolor[rgb]{0.13,0.29,0.53}{\textbf{#1}}}
\newcommand{\DataTypeTok}[1]{\textcolor[rgb]{0.13,0.29,0.53}{#1}}
\newcommand{\DecValTok}[1]{\textcolor[rgb]{0.00,0.00,0.81}{#1}}
\newcommand{\BaseNTok}[1]{\textcolor[rgb]{0.00,0.00,0.81}{#1}}
\newcommand{\FloatTok}[1]{\textcolor[rgb]{0.00,0.00,0.81}{#1}}
\newcommand{\ConstantTok}[1]{\textcolor[rgb]{0.00,0.00,0.00}{#1}}
\newcommand{\CharTok}[1]{\textcolor[rgb]{0.31,0.60,0.02}{#1}}
\newcommand{\SpecialCharTok}[1]{\textcolor[rgb]{0.00,0.00,0.00}{#1}}
\newcommand{\StringTok}[1]{\textcolor[rgb]{0.31,0.60,0.02}{#1}}
\newcommand{\VerbatimStringTok}[1]{\textcolor[rgb]{0.31,0.60,0.02}{#1}}
\newcommand{\SpecialStringTok}[1]{\textcolor[rgb]{0.31,0.60,0.02}{#1}}
\newcommand{\ImportTok}[1]{#1}
\newcommand{\CommentTok}[1]{\textcolor[rgb]{0.56,0.35,0.01}{\textit{#1}}}
\newcommand{\DocumentationTok}[1]{\textcolor[rgb]{0.56,0.35,0.01}{\textbf{\textit{#1}}}}
\newcommand{\AnnotationTok}[1]{\textcolor[rgb]{0.56,0.35,0.01}{\textbf{\textit{#1}}}}
\newcommand{\CommentVarTok}[1]{\textcolor[rgb]{0.56,0.35,0.01}{\textbf{\textit{#1}}}}
\newcommand{\OtherTok}[1]{\textcolor[rgb]{0.56,0.35,0.01}{#1}}
\newcommand{\FunctionTok}[1]{\textcolor[rgb]{0.00,0.00,0.00}{#1}}
\newcommand{\VariableTok}[1]{\textcolor[rgb]{0.00,0.00,0.00}{#1}}
\newcommand{\ControlFlowTok}[1]{\textcolor[rgb]{0.13,0.29,0.53}{\textbf{#1}}}
\newcommand{\OperatorTok}[1]{\textcolor[rgb]{0.81,0.36,0.00}{\textbf{#1}}}
\newcommand{\BuiltInTok}[1]{#1}
\newcommand{\ExtensionTok}[1]{#1}
\newcommand{\PreprocessorTok}[1]{\textcolor[rgb]{0.56,0.35,0.01}{\textit{#1}}}
\newcommand{\AttributeTok}[1]{\textcolor[rgb]{0.77,0.63,0.00}{#1}}
\newcommand{\RegionMarkerTok}[1]{#1}
\newcommand{\InformationTok}[1]{\textcolor[rgb]{0.56,0.35,0.01}{\textbf{\textit{#1}}}}
\newcommand{\WarningTok}[1]{\textcolor[rgb]{0.56,0.35,0.01}{\textbf{\textit{#1}}}}
\newcommand{\AlertTok}[1]{\textcolor[rgb]{0.94,0.16,0.16}{#1}}
\newcommand{\ErrorTok}[1]{\textcolor[rgb]{0.64,0.00,0.00}{\textbf{#1}}}
\newcommand{\NormalTok}[1]{#1}
\usepackage{longtable,booktabs}
\usepackage{graphicx,grffile}
\makeatletter
\def\maxwidth{\ifdim\Gin@nat@width>\linewidth\linewidth\else\Gin@nat@width\fi}
\def\maxheight{\ifdim\Gin@nat@height>\textheight\textheight\else\Gin@nat@height\fi}
\makeatother
% Scale images if necessary, so that they will not overflow the page
% margins by default, and it is still possible to overwrite the defaults
% using explicit options in \includegraphics[width, height, ...]{}
\setkeys{Gin}{width=\maxwidth,height=\maxheight,keepaspectratio}
\IfFileExists{parskip.sty}{%
\usepackage{parskip}
}{% else
\setlength{\parindent}{0pt}
\setlength{\parskip}{6pt plus 2pt minus 1pt}
}
\setlength{\emergencystretch}{3em}  % prevent overfull lines
\providecommand{\tightlist}{%
  \setlength{\itemsep}{0pt}\setlength{\parskip}{0pt}}
\setcounter{secnumdepth}{5}
% Redefines (sub)paragraphs to behave more like sections
\ifx\paragraph\undefined\else
\let\oldparagraph\paragraph
\renewcommand{\paragraph}[1]{\oldparagraph{#1}\mbox{}}
\fi
\ifx\subparagraph\undefined\else
\let\oldsubparagraph\subparagraph
\renewcommand{\subparagraph}[1]{\oldsubparagraph{#1}\mbox{}}
\fi

%%% Use protect on footnotes to avoid problems with footnotes in titles
\let\rmarkdownfootnote\footnote%
\def\footnote{\protect\rmarkdownfootnote}

%%% Change title format to be more compact
\usepackage{titling}

% Create subtitle command for use in maketitle
\providecommand{\subtitle}[1]{
  \posttitle{
    \begin{center}\large#1\end{center}
    }
}

\setlength{\droptitle}{-2em}

  \title{Data Science With Python}
    \pretitle{\vspace{\droptitle}\centering\huge}
  \posttitle{\par}
    \author{L A Liggett}
    \preauthor{\centering\large\emph}
  \postauthor{\par}
      \predate{\centering\large\emph}
  \postdate{\par}
    \date{2019-06-08}

\usepackage{booktabs}
\usepackage{amsthm}
\makeatletter
\def\thm@space@setup{%
  \thm@preskip=8pt plus 2pt minus 4pt
  \thm@postskip=\thm@preskip
}
\makeatother

\begin{document}
\maketitle

{
\setcounter{tocdepth}{1}
\tableofcontents
}
\chapter{Prerequisites}\label{prerequisites}

This is a \emph{sample} book written in \textbf{Markdown}. You can use
anything that Pandoc's Markdown supports, e.g., a math equation
\(a^2 + b^2 = c^2\).

The \textbf{bookdown} package can be installed from CRAN or Github:

\begin{Shaded}
\begin{Highlighting}[]
\KeywordTok{install.packages}\NormalTok{(}\StringTok{"bookdown"}\NormalTok{)}
\CommentTok{# or the development version}
\CommentTok{# devtools::install_github("rstudio/bookdown")}
\end{Highlighting}
\end{Shaded}

Remember each Rmd file contains one and only one chapter, and a chapter
is defined by the first-level heading \texttt{\#}.

To compile this example to PDF, you need XeLaTeX. You are recommended to
install TinyTeX (which includes XeLaTeX):
\url{https://yihui.name/tinytex/}.

\chapter{Introduction}\label{intro}

You can label chapter and section titles using \texttt{\{\#label\}}
after them, e.g., we can reference Chapter \ref{intro}. If you do not
manually label them, there will be automatic labels anyway, e.g.,
Chapter \ref{visualization}.

Figures and tables with captions will be placed in \texttt{figure} and
\texttt{table} environments, respectively.

And this is some other random stuff.

\begin{Shaded}
\begin{Highlighting}[]
\KeywordTok{par}\NormalTok{(}\DataTypeTok{mar =} \KeywordTok{c}\NormalTok{(}\DecValTok{4}\NormalTok{, }\DecValTok{4}\NormalTok{, .}\DecValTok{1}\NormalTok{, .}\DecValTok{1}\NormalTok{))}
\KeywordTok{plot}\NormalTok{(pressure, }\DataTypeTok{type =} \StringTok{'b'}\NormalTok{, }\DataTypeTok{pch =} \DecValTok{19}\NormalTok{)}
\end{Highlighting}
\end{Shaded}

\begin{figure}

{\centering \includegraphics[width=0.8\linewidth]{bookdown-demo_files/figure-latex/nice-fig-1} 

}

\caption{Here is a nice figure!}\label{fig:nice-fig}
\end{figure}

Reference a figure by its code chunk label with the \texttt{fig:}
prefix, e.g., see Figure \ref{fig:nice-fig}. Similarly, you can
reference tables generated from \texttt{knitr::kable()}, e.g., see Table
\ref{tab:nice-tab}.

\begin{Shaded}
\begin{Highlighting}[]
\NormalTok{knitr}\OperatorTok{::}\KeywordTok{kable}\NormalTok{(}
  \KeywordTok{head}\NormalTok{(iris, }\DecValTok{20}\NormalTok{), }\DataTypeTok{caption =} \StringTok{'Here is a nice table!'}\NormalTok{,}
  \DataTypeTok{booktabs =} \OtherTok{TRUE}
\NormalTok{)}
\end{Highlighting}
\end{Shaded}

\begin{table}[t]

\caption{\label{tab:nice-tab}Here is a nice table!}
\centering
\begin{tabular}{rrrrl}
\toprule
Sepal.Length & Sepal.Width & Petal.Length & Petal.Width & Species\\
\midrule
5.1 & 3.5 & 1.4 & 0.2 & setosa\\
4.9 & 3.0 & 1.4 & 0.2 & setosa\\
4.7 & 3.2 & 1.3 & 0.2 & setosa\\
4.6 & 3.1 & 1.5 & 0.2 & setosa\\
5.0 & 3.6 & 1.4 & 0.2 & setosa\\
\addlinespace
5.4 & 3.9 & 1.7 & 0.4 & setosa\\
4.6 & 3.4 & 1.4 & 0.3 & setosa\\
5.0 & 3.4 & 1.5 & 0.2 & setosa\\
4.4 & 2.9 & 1.4 & 0.2 & setosa\\
4.9 & 3.1 & 1.5 & 0.1 & setosa\\
\addlinespace
5.4 & 3.7 & 1.5 & 0.2 & setosa\\
4.8 & 3.4 & 1.6 & 0.2 & setosa\\
4.8 & 3.0 & 1.4 & 0.1 & setosa\\
4.3 & 3.0 & 1.1 & 0.1 & setosa\\
5.8 & 4.0 & 1.2 & 0.2 & setosa\\
\addlinespace
5.7 & 4.4 & 1.5 & 0.4 & setosa\\
5.4 & 3.9 & 1.3 & 0.4 & setosa\\
5.1 & 3.5 & 1.4 & 0.3 & setosa\\
5.7 & 3.8 & 1.7 & 0.3 & setosa\\
5.1 & 3.8 & 1.5 & 0.3 & setosa\\
\bottomrule
\end{tabular}
\end{table}

\begin{Shaded}
\begin{Highlighting}[]
\NormalTok{knitr}\OperatorTok{::}\KeywordTok{include_graphics}\NormalTok{(}\KeywordTok{rep}\NormalTok{(}\StringTok{"knit-logo.png"}\NormalTok{, }\DecValTok{3}\NormalTok{))}
\end{Highlighting}
\end{Shaded}

\includegraphics{knit-logo.png} \includegraphics{knit-logo.png}
\includegraphics{knit-logo.png}

\begin{Shaded}
\begin{Highlighting}[]
\NormalTok{knitr}\OperatorTok{::}\KeywordTok{include_app}\NormalTok{(}\StringTok{"https://yihui.shinyapps.io/miniUI/"}\NormalTok{, }
                     \DataTypeTok{height =} \StringTok{"600px"}\NormalTok{)}
\end{Highlighting}
\end{Shaded}

\begin{Shaded}
\begin{Highlighting}[]
\ImportTok{import}\NormalTok{ pandas }\ImportTok{as}\NormalTok{ pd}
\NormalTok{x }\OperatorTok{=} \StringTok{'hello, python world!'}
\BuiltInTok{print}\NormalTok{(x.split(}\StringTok{' '}\NormalTok{))}
\end{Highlighting}
\end{Shaded}

\begin{verbatim}
## ['hello,', 'python', 'world!']
\end{verbatim}

You can write citations, too. For example, we are using the
\textbf{bookdown} package \citep{R-bookdown} in this sample book, which
was built on top of R Markdown and \textbf{knitr} \citep{xie2015}.

\chapter{JupyterLab}\label{jupyter}

Here is a simple template that I use that controls a couple useful
things when starting a new notebook.

\begin{Shaded}
\begin{Highlighting}[]
\ImportTok{import}\NormalTok{ sys}
\NormalTok{sys.path.append(}\StringTok{'../util'}\NormalTok{)}

\OperatorTok\NormalTok{autoreload }\DecValTok{2}

\ImportTok{from}\NormalTok{ util }\ImportTok{import} \OperatorTok{*}
\ImportTok{import}\NormalTok{ numpy }\ImportTok{as}\NormalTok{ np                  }
\ImportTok{import}\NormalTok{ pandas }\ImportTok{as}\NormalTok{ pd                 }
\ImportTok{from}\NormalTok{ matplotlib }\ImportTok{import}\NormalTok{ pyplot }\ImportTok{as}\NormalTok{ plt}
\ImportTok{import}\NormalTok{ seaborn }\ImportTok{as}\NormalTok{ sns}

\NormalTok{sns.set_palette(}\StringTok{'pastel'}\NormalTok{)}
\NormalTok{sns.set_style(}\StringTok{'ticks'}\NormalTok{)}
\NormalTok{sns.set_context(}\StringTok{'paper'}\NormalTok{, font_scale}\OperatorTok{=}\DecValTok{1}\NormalTok{)}
\end{Highlighting}
\end{Shaded}

It is often convenient to have a notebook automatically refresh the
imported libraries so that they can be modified while working on a
JupyterLab notebook.

\begin{Shaded}
\begin{Highlighting}[]
\OperatorTok\NormalTok{autoreload }\DecValTok{2}
\end{Highlighting}
\end{Shaded}

To allow directory organization, dependcies can be separated into
different directories and imported into a jupyter notebook using the
following import statement.

\begin{Shaded}
\begin{Highlighting}[]
\ImportTok{import}\NormalTok{ sys}
\NormalTok{sys.path.append(}\StringTok{'../util'}\NormalTok{)}
\end{Highlighting}
\end{Shaded}

A table of contents can be created to refer to each of the headers
throughout a notebook in html format. The code is below (Obviously needs
to be simplified.)

\begin{Shaded}
\begin{Highlighting}[]
\OperatorTok{<}\NormalTok{h1}\OperatorTok{>}\NormalTok{Table of Contents}\OperatorTok{<}\NormalTok{span }\KeywordTok{class}\OperatorTok{=}\StringTok{"tocSkip"}\OperatorTok{></}\NormalTok{span}\OperatorTok{></}\NormalTok{h1}\OperatorTok{>}
\OperatorTok{<}\NormalTok{div }\KeywordTok{class}\OperatorTok{=}\StringTok{"toc"}\OperatorTok{>}
    \OperatorTok{<}\NormalTok{ul }\KeywordTok{class}\OperatorTok{=}\StringTok{"toc-item"}\OperatorTok{>}
    \OperatorTok{<}\NormalTok{li}\OperatorTok{>}
        \OperatorTok{<}\NormalTok{span}\OperatorTok{><}\NormalTok{a href}\OperatorTok{=}\StringTok{"#Python-Setup"}\NormalTok{ data}\OperatorTok{-}\NormalTok{toc}\OperatorTok{-}\NormalTok{modified}\OperatorTok{-}\BuiltInTok{id}\OperatorTok{=}\StringTok{"Python-Setup-1"}\OperatorTok{><}\NormalTok{span }\KeywordTok{class}\OperatorTok{=}\StringTok{"toc-item-num"}\OperatorTok{>}\DecValTok{1}\OperatorTok{&}\NormalTok{nbsp}\OperatorTok{;&}\NormalTok{nbsp}\OperatorTok{;</}\NormalTok{span}\OperatorTok{>}\NormalTok{Python Setup}\OperatorTok{</}\NormalTok{a}\OperatorTok{></}\NormalTok{span}\OperatorTok{>}
        \OperatorTok{<}\NormalTok{ul }\KeywordTok{class}\OperatorTok{=}\StringTok{"toc-item"}\OperatorTok{>}
    \OperatorTok{<}\NormalTok{li}\OperatorTok{>}
        \OperatorTok{<}\NormalTok{span}\OperatorTok{><}\NormalTok{a href}\OperatorTok{=}\StringTok{"#Change-the-width-of-the-page"}\NormalTok{ data}\OperatorTok{-}\NormalTok{toc}\OperatorTok{-}\NormalTok{modified}\OperatorTok{-}\BuiltInTok{id}\OperatorTok{=}\StringTok{"Change-the-width-of-the-page-1.1"}\OperatorTok{><}\NormalTok{span }\KeywordTok{class}\OperatorTok{=}\StringTok{"toc-item-num"}\OperatorTok{>}\FloatTok{1.1}\OperatorTok{&}\NormalTok{nbsp}\OperatorTok{;&}\NormalTok{nbsp}\OperatorTok{;</}\NormalTok{span}\OperatorTok{>}\NormalTok{Change the width of the page}\OperatorTok{</}\NormalTok{a}\OperatorTok{></}\NormalTok{span}\OperatorTok{></}\NormalTok{li}\OperatorTok{>}
        \OperatorTok{<}\NormalTok{li}\OperatorTok{>}
            \OperatorTok{<}\NormalTok{span}\OperatorTok{><}\NormalTok{a href}\OperatorTok{=}\StringTok{"#Import-packages"}\NormalTok{ data}\OperatorTok{-}\NormalTok{toc}\OperatorTok{-}\NormalTok{modified}\OperatorTok{-}\BuiltInTok{id}\OperatorTok{=}\StringTok{"Import-packages-1.2"}\OperatorTok{><}\NormalTok{span }\KeywordTok{class}\OperatorTok{=}\StringTok{"toc-item-num"}\OperatorTok{>}\FloatTok{1.2}\OperatorTok{&}\NormalTok{nbsp}\OperatorTok{;&}\NormalTok{nbsp}\OperatorTok{;</}\NormalTok{span}\OperatorTok{>}\NormalTok{Import packages}\OperatorTok{</}\NormalTok{a}\OperatorTok{></}\NormalTok{span}\OperatorTok{></}\NormalTok{li}\OperatorTok{>}
        \OperatorTok{</}\NormalTok{ul}\OperatorTok{>}
    \OperatorTok{</}\NormalTok{li}\OperatorTok{>}
    \OperatorTok{<}\NormalTok{li}\OperatorTok{>}
        \OperatorTok{<}\NormalTok{span}\OperatorTok{><}\NormalTok{a href}\OperatorTok{=}\StringTok{"#Colours"}\NormalTok{ data}\OperatorTok{-}\NormalTok{toc}\OperatorTok{-}\NormalTok{modified}\OperatorTok{-}\BuiltInTok{id}\OperatorTok{=}\StringTok{"Colours-2"}\OperatorTok{><}\NormalTok{span }\KeywordTok{class}\OperatorTok{=}\StringTok{"toc-item-num"}\OperatorTok{>}\DecValTok{2}\OperatorTok{&}\NormalTok{nbsp}\OperatorTok{;&}\NormalTok{nbsp}\OperatorTok{;</}\NormalTok{span}\OperatorTok{>}\NormalTok{Colours}\OperatorTok{</}\NormalTok{a}\OperatorTok{></}\NormalTok{span}\OperatorTok{>}
        \OperatorTok{<}\NormalTok{ul }\KeywordTok{class}\OperatorTok{=}\StringTok{"toc-item"}\OperatorTok{>}
    \OperatorTok{<}\NormalTok{li}\OperatorTok{><}\NormalTok{span}\OperatorTok{><}\NormalTok{a href}\OperatorTok{=}\StringTok{"#Colour-line-graph"}\NormalTok{ data}\OperatorTok{-}\NormalTok{toc}\OperatorTok{-}\NormalTok{modified}\OperatorTok{-}\BuiltInTok{id}\OperatorTok{=}\StringTok{"Colour-line-graph-2.1"}\OperatorTok{><}\NormalTok{span }\KeywordTok{class}\OperatorTok{=}\StringTok{"toc-item-num"}\OperatorTok{>}\FloatTok{2.1}\OperatorTok{&}\NormalTok{nbsp}\OperatorTok{;&}\NormalTok{nbsp}\OperatorTok{;</}\NormalTok{span}\OperatorTok{>}\NormalTok{Colour line graph}\OperatorTok{</}\NormalTok{a}\OperatorTok{></}\NormalTok{span}\OperatorTok{></}\NormalTok{li}\OperatorTok{>}
        \OperatorTok{</}\NormalTok{ul}\OperatorTok{>}
    \OperatorTok{</}\NormalTok{li}\OperatorTok{>}
    \OperatorTok{<}\NormalTok{li}\OperatorTok{><}\NormalTok{span}\OperatorTok{><}\NormalTok{a href}\OperatorTok{=}\StringTok{"#Totals-for-studies"}\NormalTok{ data}\OperatorTok{-}\NormalTok{toc}\OperatorTok{-}\NormalTok{modified}\OperatorTok{-}\BuiltInTok{id}\OperatorTok{=}\StringTok{"Totals-for-studies-3"}\OperatorTok{><}\NormalTok{span }\KeywordTok{class}\OperatorTok{=}\StringTok{"toc-item-num"}\OperatorTok{>}\DecValTok{3}\OperatorTok{&}\NormalTok{nbsp}\OperatorTok{;&}\NormalTok{nbsp}\OperatorTok{;</}\NormalTok{span}\OperatorTok{>}\NormalTok{Totals }\ControlFlowTok{for}\NormalTok{ studies}\OperatorTok{</}\NormalTok{a}\OperatorTok{></}\NormalTok{span}\OperatorTok{></}\NormalTok{li}\OperatorTok{>}
    \OperatorTok{<}\NormalTok{li}\OperatorTok{><}\NormalTok{span}\OperatorTok{><}\NormalTok{a href}\OperatorTok{=}\StringTok{"#Functions-for-calculating-trinucleotide-context-specific-mutation-rates"}\NormalTok{ data}\OperatorTok{-}\NormalTok{toc}\OperatorTok{-}\NormalTok{modified}\OperatorTok{-}\BuiltInTok{id}\OperatorTok{=}\StringTok{"Functions-for-calculating-trinucleotide-context-specific-mutation-rates-4"}\OperatorTok{><}\NormalTok{span }\KeywordTok{class}\OperatorTok{=}\StringTok{"toc-item-num"}\OperatorTok{>}\DecValTok{4}\OperatorTok{&}\NormalTok{nbsp}\OperatorTok{;&}\NormalTok{nbsp}\OperatorTok{;</}\NormalTok{span}\OperatorTok{>}\NormalTok{Functions }\ControlFlowTok{for}\NormalTok{ calculating trinucleotide}\OperatorTok{-}\NormalTok{context specific mutation rates}\OperatorTok{</}\NormalTok{a}\OperatorTok{></}\NormalTok{span}\OperatorTok{>}
        \OperatorTok{<}\NormalTok{ul }\KeywordTok{class}\OperatorTok{=}\StringTok{"toc-item"}\OperatorTok{>}
        \OperatorTok{<}\NormalTok{li}\OperatorTok{><}\NormalTok{span}\OperatorTok{><}\NormalTok{a href}\OperatorTok{=}\StringTok{"#Calculating-mutation-rates-for-individual-variants"}\NormalTok{ data}\OperatorTok{-}\NormalTok{toc}\OperatorTok{-}\NormalTok{modified}\OperatorTok{-}\BuiltInTok{id}\OperatorTok{=}\StringTok{"Calculating-mutation-rates-for-individual-variants-4.1"}\OperatorTok{><}\NormalTok{span }\KeywordTok{class}\OperatorTok{=}\StringTok{"toc-item-num"}\OperatorTok{>}\FloatTok{4.1}\OperatorTok{&}\NormalTok{nbsp}\OperatorTok{;&}\NormalTok{nbsp}\OperatorTok{;</}\NormalTok{span}\OperatorTok{>}\NormalTok{Calculating mutation rates }\ControlFlowTok{for}\NormalTok{ individual variants}\OperatorTok{</}\NormalTok{a}\OperatorTok{></}\NormalTok{span}\OperatorTok{>}
            \OperatorTok{<}\NormalTok{ul }\KeywordTok{class}\OperatorTok{=}\StringTok{"toc-item"}\OperatorTok{>}
        \OperatorTok{<}\NormalTok{li}\OperatorTok{><}\NormalTok{span}\OperatorTok{><}\NormalTok{a href}\OperatorTok{=}\StringTok{"#DNMT3A"}\NormalTok{ data}\OperatorTok{-}\NormalTok{toc}\OperatorTok{-}\NormalTok{modified}\OperatorTok{-}\BuiltInTok{id}\OperatorTok{=}\StringTok{"DNMT3A-4.1.1"}\OperatorTok{><}\NormalTok{span }\KeywordTok{class}\OperatorTok{=}\StringTok{"toc-item-num"}\OperatorTok{>}\FloatTok{4.1}\NormalTok{.}\DecValTok{1}\OperatorTok{&}\NormalTok{nbsp}\OperatorTok{;&}\NormalTok{nbsp}\OperatorTok{;</}\NormalTok{span}\OperatorTok{>}\NormalTok{DNMT3A}\OperatorTok{</}\NormalTok{a}\OperatorTok{></}\NormalTok{span}\OperatorTok{></}\NormalTok{li}\OperatorTok{>}
            \OperatorTok{<}\NormalTok{li}\OperatorTok{><}\NormalTok{span}\OperatorTok{><}\NormalTok{a href}\OperatorTok{=}\StringTok{"#TET2"}\NormalTok{ data}\OperatorTok{-}\NormalTok{toc}\OperatorTok{-}\NormalTok{modified}\OperatorTok{-}\BuiltInTok{id}\OperatorTok{=}\StringTok{"TET2-4.1.2"}\OperatorTok{><}\NormalTok{span }\KeywordTok{class}\OperatorTok{=}\StringTok{"toc-item-num"}\OperatorTok{>}\FloatTok{4.1}\NormalTok{.}\DecValTok{2}\OperatorTok{&}\NormalTok{nbsp}\OperatorTok{;&}\NormalTok{nbsp}\OperatorTok{;</}\NormalTok{span}\OperatorTok{>}\NormalTok{TET2}\OperatorTok{</}\NormalTok{a}\OperatorTok{></}\NormalTok{span}\OperatorTok{></}\NormalTok{li}\OperatorTok{>}
            \OperatorTok{<}\NormalTok{li}\OperatorTok{><}\NormalTok{span}\OperatorTok{><}\NormalTok{a href}\OperatorTok{=}\StringTok{"#ASXL1"}\NormalTok{ data}\OperatorTok{-}\NormalTok{toc}\OperatorTok{-}\NormalTok{modified}\OperatorTok{-}\BuiltInTok{id}\OperatorTok{=}\StringTok{"ASXL1-4.1.3"}\OperatorTok{><}\NormalTok{span }\KeywordTok{class}\OperatorTok{=}\StringTok{"toc-item-num"}\OperatorTok{>}\FloatTok{4.1}\NormalTok{.}\DecValTok{3}\OperatorTok{&}\NormalTok{nbsp}\OperatorTok{;&}\NormalTok{nbsp}\OperatorTok{;</}\NormalTok{span}\OperatorTok{>}\NormalTok{ASXL1}\OperatorTok{</}\NormalTok{a}\OperatorTok{></}\NormalTok{span}\OperatorTok{></}\NormalTok{li}\OperatorTok{>}
            \OperatorTok{<}\NormalTok{li}\OperatorTok{><}\NormalTok{span}\OperatorTok{><}\NormalTok{a href}\OperatorTok{=}\StringTok{"#TP53"}\NormalTok{ data}\OperatorTok{-}\NormalTok{toc}\OperatorTok{-}\NormalTok{modified}\OperatorTok{-}\BuiltInTok{id}\OperatorTok{=}\StringTok{"TP53-4.1.4"}\OperatorTok{><}\NormalTok{span }\KeywordTok{class}\OperatorTok{=}\StringTok{"toc-item-num"}\OperatorTok{>}\FloatTok{4.1}\NormalTok{.}\DecValTok{4}\OperatorTok{&}\NormalTok{nbsp}\OperatorTok{;&}\NormalTok{nbsp}\OperatorTok{;</}\NormalTok{span}\OperatorTok{>}\NormalTok{TP53}\OperatorTok{</}\NormalTok{a}\OperatorTok{></}\NormalTok{span}\OperatorTok{></}\NormalTok{li}\OperatorTok{></}\NormalTok{ul}\OperatorTok{></}\NormalTok{li}\OperatorTok{>}
        \OperatorTok{<}\NormalTok{li}\OperatorTok{><}\NormalTok{span}\OperatorTok{><}\NormalTok{a href}\OperatorTok{=}\StringTok{"#Calculating-mutation-rates-from-a-.csv-file-of-variants"}\NormalTok{ data}\OperatorTok{-}\NormalTok{toc}\OperatorTok{-}\NormalTok{modified}\OperatorTok{-}\BuiltInTok{id}\OperatorTok{=}\StringTok{"Calculating-mutation-rates-from-a-.csv-file-of-variants-4.2"}\OperatorTok{><}\NormalTok{span }\KeywordTok{class}\OperatorTok{=}\StringTok{"toc-item-num"}\OperatorTok{>}\FloatTok{4.2}\OperatorTok{&}\NormalTok{nbsp}\OperatorTok{;&}\NormalTok{nbsp}\OperatorTok{;</}\NormalTok{span}\OperatorTok{>}\NormalTok{Calculating mutation rates }\ImportTok{from}\NormalTok{ a .csv }\BuiltInTok{file}\NormalTok{ of variants}\OperatorTok{</}\NormalTok{a}\OperatorTok{></}\NormalTok{span}\OperatorTok{>}
            \OperatorTok{<}\NormalTok{ul }\KeywordTok{class}\OperatorTok{=}\StringTok{"toc-item"}\OperatorTok{>}
            \OperatorTok{<}\NormalTok{li}\OperatorTok{><}\NormalTok{span}\OperatorTok{><}\NormalTok{a href}\OperatorTok{=}\StringTok{"#DNMT3A"}\NormalTok{ data}\OperatorTok{-}\NormalTok{toc}\OperatorTok{-}\NormalTok{modified}\OperatorTok{-}\BuiltInTok{id}\OperatorTok{=}\StringTok{"DNMT3A-4.2.1"}\OperatorTok{><}\NormalTok{span }\KeywordTok{class}\OperatorTok{=}\StringTok{"toc-item-num"}\OperatorTok{>}\FloatTok{4.2}\NormalTok{.}\DecValTok{1}\OperatorTok{&}\NormalTok{nbsp}\OperatorTok{;&}\NormalTok{nbsp}\OperatorTok{;</}\NormalTok{span}\OperatorTok{>}\NormalTok{DNMT3A}\OperatorTok{</}\NormalTok{a}\OperatorTok{></}\NormalTok{span}\OperatorTok{></}\NormalTok{li}\OperatorTok{>}
            \OperatorTok{<}\NormalTok{li}\OperatorTok{><}\NormalTok{span}\OperatorTok{><}\NormalTok{a href}\OperatorTok{=}\StringTok{"#TET2"}\NormalTok{ data}\OperatorTok{-}\NormalTok{toc}\OperatorTok{-}\NormalTok{modified}\OperatorTok{-}\BuiltInTok{id}\OperatorTok{=}\StringTok{"TET2-4.2.2"}\OperatorTok{><}\NormalTok{span }\KeywordTok{class}\OperatorTok{=}\StringTok{"toc-item-num"}\OperatorTok{>}\FloatTok{4.2}\NormalTok{.}\DecValTok{2}\OperatorTok{&}\NormalTok{nbsp}\OperatorTok{;&}\NormalTok{nbsp}\OperatorTok{;</}\NormalTok{span}\OperatorTok{>}\NormalTok{TET2}\OperatorTok{</}\NormalTok{a}\OperatorTok{></}\NormalTok{span}\OperatorTok{></}\NormalTok{li}\OperatorTok{>}
            \OperatorTok{<}\NormalTok{li}\OperatorTok{><}\NormalTok{span}\OperatorTok{><}\NormalTok{a href}\OperatorTok{=}\StringTok{"#ASXL1"}\NormalTok{ data}\OperatorTok{-}\NormalTok{toc}\OperatorTok{-}\NormalTok{modified}\OperatorTok{-}\BuiltInTok{id}\OperatorTok{=}\StringTok{"ASXL1-4.2.3"}\OperatorTok{><}\NormalTok{span }\KeywordTok{class}\OperatorTok{=}\StringTok{"toc-item-num"}\OperatorTok{>}\FloatTok{4.2}\NormalTok{.}\DecValTok{3}\OperatorTok{&}\NormalTok{nbsp}\OperatorTok{;&}\NormalTok{nbsp}\OperatorTok{;</}\NormalTok{span}\OperatorTok{>}\NormalTok{ASXL1}\OperatorTok{</}\NormalTok{a}\OperatorTok{></}\NormalTok{span}\OperatorTok{></}\NormalTok{li}\OperatorTok{>}
            \OperatorTok{<}\NormalTok{li}\OperatorTok{><}\NormalTok{span}\OperatorTok{><}\NormalTok{a href}\OperatorTok{=}\StringTok{"#TP53"}\NormalTok{ data}\OperatorTok{-}\NormalTok{toc}\OperatorTok{-}\NormalTok{modified}\OperatorTok{-}\BuiltInTok{id}\OperatorTok{=}\StringTok{"TP53-4.2.4"}\OperatorTok{><}\NormalTok{span }\KeywordTok{class}\OperatorTok{=}\StringTok{"toc-item-num"}\OperatorTok{>}\FloatTok{4.2}\NormalTok{.}\DecValTok{4}\OperatorTok{&}\NormalTok{nbsp}\OperatorTok{;&}\NormalTok{nbsp}\OperatorTok{;</}\NormalTok{span}\OperatorTok{>}\NormalTok{TP53}\OperatorTok{</}\NormalTok{a}\OperatorTok{></}\NormalTok{span}\OperatorTok{></}\NormalTok{li}\OperatorTok{></}\NormalTok{ul}\OperatorTok{></}\NormalTok{li}\OperatorTok{>}
        \OperatorTok{<}\NormalTok{li}\OperatorTok{><}\NormalTok{span}\OperatorTok{><}\NormalTok{a href}\OperatorTok{=}\StringTok{"#Calculating-mutation-rates-from-a-list-of-variants"}\NormalTok{ data}\OperatorTok{-}\NormalTok{toc}\OperatorTok{-}\NormalTok{modified}\OperatorTok{-}\BuiltInTok{id}\OperatorTok{=}\StringTok{"Calculating-mutation-rates-from-a-list-of-variants-4.3"}\OperatorTok{><}\NormalTok{span }\KeywordTok{class}\OperatorTok{=}\StringTok{"toc-item-num"}\OperatorTok{>}\FloatTok{4.3}\OperatorTok{&}\NormalTok{nbsp}\OperatorTok{;&}\NormalTok{nbsp}\OperatorTok{;</}\NormalTok{span}\OperatorTok{>}\NormalTok{Calculating mutation rates }\ImportTok{from}\NormalTok{ a }\BuiltInTok{list}\NormalTok{ of variants}\OperatorTok{</}\NormalTok{a}\OperatorTok{></}\NormalTok{span}\OperatorTok{>}
            \OperatorTok{<}\NormalTok{ul }\KeywordTok{class}\OperatorTok{=}\StringTok{"toc-item"}\OperatorTok{>}
            \OperatorTok{<}\NormalTok{li}\OperatorTok{><}\NormalTok{span}\OperatorTok{><}\NormalTok{a href}\OperatorTok{=}\StringTok{"#DNMT3A"}\NormalTok{ data}\OperatorTok{-}\NormalTok{toc}\OperatorTok{-}\NormalTok{modified}\OperatorTok{-}\BuiltInTok{id}\OperatorTok{=}\StringTok{"DNMT3A-4.3.1"}\OperatorTok{><}\NormalTok{span }\KeywordTok{class}\OperatorTok{=}\StringTok{"toc-item-num"}\OperatorTok{>}\FloatTok{4.3}\NormalTok{.}\DecValTok{1}\OperatorTok{&}\NormalTok{nbsp}\OperatorTok{;&}\NormalTok{nbsp}\OperatorTok{;</}\NormalTok{span}\OperatorTok{>}\NormalTok{DNMT3A}\OperatorTok{</}\NormalTok{a}\OperatorTok{></}\NormalTok{span}\OperatorTok{></}\NormalTok{li}\OperatorTok{>}
        \OperatorTok{<}\NormalTok{li}\OperatorTok{><}\NormalTok{span}\OperatorTok{><}\NormalTok{a href}\OperatorTok{=}\StringTok{"#TET2"}\NormalTok{ data}\OperatorTok{-}\NormalTok{toc}\OperatorTok{-}\NormalTok{modified}\OperatorTok{-}\BuiltInTok{id}\OperatorTok{=}\StringTok{"TET2-4.3.2"}\OperatorTok{><}\NormalTok{span }\KeywordTok{class}\OperatorTok{=}\StringTok{"toc-item-num"}\OperatorTok{>}\FloatTok{4.3}\NormalTok{.}\DecValTok{2}\OperatorTok{&}\NormalTok{nbsp}\OperatorTok{;&}\NormalTok{nbsp}\OperatorTok{;</}\NormalTok{span}\OperatorTok{>}\NormalTok{TET2}\OperatorTok{</}\NormalTok{a}\OperatorTok{></}\NormalTok{span}\OperatorTok{></}\NormalTok{li}\OperatorTok{>}
            \OperatorTok{<}\NormalTok{li}\OperatorTok{><}\NormalTok{span}\OperatorTok{><}\NormalTok{a href}\OperatorTok{=}\StringTok{"#ASXL1"}\NormalTok{ data}\OperatorTok{-}\NormalTok{toc}\OperatorTok{-}\NormalTok{modified}\OperatorTok{-}\BuiltInTok{id}\OperatorTok{=}\StringTok{"ASXL1-4.3.3"}\OperatorTok{><}\NormalTok{span }\KeywordTok{class}\OperatorTok{=}\StringTok{"toc-item-num"}\OperatorTok{>}\FloatTok{4.3}\NormalTok{.}\DecValTok{3}\OperatorTok{&}\NormalTok{nbsp}\OperatorTok{;&}\NormalTok{nbsp}\OperatorTok{;</}\NormalTok{span}\OperatorTok{>}\NormalTok{ASXL1}\OperatorTok{</}\NormalTok{a}\OperatorTok{></}\NormalTok{span}\OperatorTok{></}\NormalTok{li}\OperatorTok{>}
            \OperatorTok{<}\NormalTok{li}\OperatorTok{><}\NormalTok{span}\OperatorTok{><}\NormalTok{a href}\OperatorTok{=}\StringTok{"#TP53"}\NormalTok{ data}\OperatorTok{-}\NormalTok{toc}\OperatorTok{-}\NormalTok{modified}\OperatorTok{-}\BuiltInTok{id}\OperatorTok{=}\StringTok{"TP53-4.3.4"}\OperatorTok{><}\NormalTok{span }\KeywordTok{class}\OperatorTok{=}\StringTok{"toc-item-num"}\OperatorTok{>}\FloatTok{4.3}\NormalTok{.}\DecValTok{4}\OperatorTok{&}\NormalTok{nbsp}\OperatorTok{;&}\NormalTok{nbsp}\OperatorTok{;</}\NormalTok{span}\OperatorTok{>}\NormalTok{TP53}\OperatorTok{</}\NormalTok{a}\OperatorTok{></}\NormalTok{span}\OperatorTok{></}\NormalTok{li}\OperatorTok{></}\NormalTok{ul}\OperatorTok{></}\NormalTok{li}\OperatorTok{></}\NormalTok{ul}\OperatorTok{></}\NormalTok{li}\OperatorTok{>}
    \OperatorTok{<}\NormalTok{li}\OperatorTok{><}\NormalTok{span}\OperatorTok{><}\NormalTok{a href}\OperatorTok{=}\StringTok{"#Lists-of-variants-targeted-by-each-study"}\NormalTok{ data}\OperatorTok{-}\NormalTok{toc}\OperatorTok{-}\NormalTok{modified}\OperatorTok{-}\BuiltInTok{id}\OperatorTok{=}\StringTok{"Lists-of-variants-targeted-by-each-study-5"}\OperatorTok{><}\NormalTok{span }\KeywordTok{class}\OperatorTok{=}\StringTok{"toc-item-num"}\OperatorTok{>}\DecValTok{5}\OperatorTok{&}\NormalTok{nbsp}\OperatorTok{;&}\NormalTok{nbsp}\OperatorTok{;</}\NormalTok{span}\OperatorTok{>}\NormalTok{Lists of variants targeted by each study}\OperatorTok{</}\NormalTok{a}\OperatorTok{></}\NormalTok{span}\OperatorTok{><}\NormalTok{ul }\KeywordTok{class}\OperatorTok{=}\StringTok{"toc-item"}\OperatorTok{><}\NormalTok{li}\OperatorTok{><}\NormalTok{span}\OperatorTok{><}\NormalTok{a href}\OperatorTok{=}\StringTok{"#Jaiswal-2014"}\NormalTok{ data}\OperatorTok{-}\NormalTok{toc}\OperatorTok{-}\NormalTok{modified}\OperatorTok{-}\BuiltInTok{id}\OperatorTok{=}\StringTok{"Jaiswal-2014-5.1"}\OperatorTok{><}\NormalTok{span }\KeywordTok{class}\OperatorTok{=}\StringTok{"toc-item-num"}\OperatorTok{>}\FloatTok{5.1}\OperatorTok{&}\NormalTok{nbsp}\OperatorTok{;&}\NormalTok{nbsp}\OperatorTok{;</}\NormalTok{span}\OperatorTok{>}\NormalTok{Jaiswal }\DecValTok{2014}\OperatorTok{</}\NormalTok{a}\OperatorTok{></}\NormalTok{span}\OperatorTok{></}\NormalTok{li}\OperatorTok{>}
        \OperatorTok{<}\NormalTok{li}\OperatorTok{><}\NormalTok{span}\OperatorTok{><}\NormalTok{a href}\OperatorTok{=}\StringTok{"#Genovese-2014"}\NormalTok{ data}\OperatorTok{-}\NormalTok{toc}\OperatorTok{-}\NormalTok{modified}\OperatorTok{-}\BuiltInTok{id}\OperatorTok{=}\StringTok{"Genovese-2014-5.2"}\OperatorTok{><}\NormalTok{span }\KeywordTok{class}\OperatorTok{=}\StringTok{"toc-item-num"}\OperatorTok{>}\FloatTok{5.2}\OperatorTok{&}\NormalTok{nbsp}\OperatorTok{;&}\NormalTok{nbsp}\OperatorTok{;</}\NormalTok{span}\OperatorTok{>}\NormalTok{Genovese }\DecValTok{2014}\OperatorTok{</}\NormalTok{a}\OperatorTok{></}\NormalTok{span}\OperatorTok{></}\NormalTok{li}\OperatorTok{>}
        \OperatorTok{<}\NormalTok{li}\OperatorTok{><}\NormalTok{span}\OperatorTok{><}\NormalTok{a href}\OperatorTok{=}\StringTok{"#McKerrel-2015"}\NormalTok{ data}\OperatorTok{-}\NormalTok{toc}\OperatorTok{-}\NormalTok{modified}\OperatorTok{-}\BuiltInTok{id}\OperatorTok{=}\StringTok{"McKerrel-2015-5.3"}\OperatorTok{><}\NormalTok{span }\KeywordTok{class}\OperatorTok{=}\StringTok{"toc-item-num"}\OperatorTok{>}\FloatTok{5.3}\OperatorTok{&}\NormalTok{nbsp}\OperatorTok{;&}\NormalTok{nbsp}\OperatorTok{;</}\NormalTok{span}\OperatorTok{>}\NormalTok{McKerrel }\DecValTok{2015}\OperatorTok{</}\NormalTok{a}\OperatorTok{></}\NormalTok{span}\OperatorTok{></}\NormalTok{li}\OperatorTok{>}
        \OperatorTok{<}\NormalTok{li}\OperatorTok{><}\NormalTok{span}\OperatorTok{><}\NormalTok{a href}\OperatorTok{=}\StringTok{"#Zink-2017"}\NormalTok{ data}\OperatorTok{-}\NormalTok{toc}\OperatorTok{-}\NormalTok{modified}\OperatorTok{-}\BuiltInTok{id}\OperatorTok{=}\StringTok{"Zink-2017-5.4"}\OperatorTok{><}\NormalTok{span }\KeywordTok{class}\OperatorTok{=}\StringTok{"toc-item-num"}\OperatorTok{>}\FloatTok{5.4}\OperatorTok{&}\NormalTok{nbsp}\OperatorTok{;&}\NormalTok{nbsp}\OperatorTok{;</}\NormalTok{span}\OperatorTok{>}\NormalTok{Zink }\DecValTok{2017}\OperatorTok{</}\NormalTok{a}\OperatorTok{></}\NormalTok{span}\OperatorTok{></}\NormalTok{li}\OperatorTok{>}
        \OperatorTok{<}\NormalTok{li}\OperatorTok{><}\NormalTok{span}\OperatorTok{><}\NormalTok{a href}\OperatorTok{=}\StringTok{"#Coombs-2017"}\NormalTok{ data}\OperatorTok{-}\NormalTok{toc}\OperatorTok{-}\NormalTok{modified}\OperatorTok{-}\BuiltInTok{id}\OperatorTok{=}\StringTok{"Coombs-2017-5.5"}\OperatorTok{><}\NormalTok{span }\KeywordTok{class}\OperatorTok{=}\StringTok{"toc-item-num"}\OperatorTok{>}\FloatTok{5.5}\OperatorTok{&}\NormalTok{nbsp}\OperatorTok{;&}\NormalTok{nbsp}\OperatorTok{;</}\NormalTok{span}\OperatorTok{>}\NormalTok{Coombs }\DecValTok{2017}\OperatorTok{</}\NormalTok{a}\OperatorTok{></}\NormalTok{span}\OperatorTok{></}\NormalTok{li}\OperatorTok{>}
        \OperatorTok{<}\NormalTok{li}\OperatorTok{><}\NormalTok{span}\OperatorTok{><}\NormalTok{a href}\OperatorTok{=}\StringTok{"#Young-2016-&amp;-2019"}\NormalTok{ data}\OperatorTok{-}\NormalTok{toc}\OperatorTok{-}\NormalTok{modified}\OperatorTok{-}\BuiltInTok{id}\OperatorTok{=}\StringTok{"Young-2016-&amp;-2019-5.6"}\OperatorTok{><}\NormalTok{span }\KeywordTok{class}\OperatorTok{=}\StringTok{"toc-item-num"}\OperatorTok{>}\FloatTok{5.6}\OperatorTok{&}\NormalTok{nbsp}\OperatorTok{;&}\NormalTok{nbsp}\OperatorTok{;</}\NormalTok{span}\OperatorTok{>}\NormalTok{Young }\DecValTok{2016} \OperatorTok{&}\NormalTok{amp}\OperatorTok{;} \DecValTok{2019}\OperatorTok{</}\NormalTok{a}\OperatorTok{></}\NormalTok{span}\OperatorTok{></}\NormalTok{li}\OperatorTok{>}
        \OperatorTok{<}\NormalTok{li}\OperatorTok{><}\NormalTok{span}\OperatorTok{><}\NormalTok{a href}\OperatorTok{=}\StringTok{"#Desai-2018"}\NormalTok{ data}\OperatorTok{-}\NormalTok{toc}\OperatorTok{-}\NormalTok{modified}\OperatorTok{-}\BuiltInTok{id}\OperatorTok{=}\StringTok{"Desai-2018-5.7"}\OperatorTok{><}\NormalTok{span }\KeywordTok{class}\OperatorTok{=}\StringTok{"toc-item-num"}\OperatorTok{>}\FloatTok{5.7}\OperatorTok{&}\NormalTok{nbsp}\OperatorTok{;&}\NormalTok{nbsp}\OperatorTok{;</}\NormalTok{span}\OperatorTok{>}\NormalTok{Desai }\DecValTok{2018}\OperatorTok{</}\NormalTok{a}\OperatorTok{></}\NormalTok{span}\OperatorTok{></}\NormalTok{li}\OperatorTok{>}
        \OperatorTok{<}\NormalTok{li}\OperatorTok{><}\NormalTok{span}\OperatorTok{><}\NormalTok{a href}\OperatorTok{=}\StringTok{"#Acuna-Hidalgo-2017"}\NormalTok{ data}\OperatorTok{-}\NormalTok{toc}\OperatorTok{-}\NormalTok{modified}\OperatorTok{-}\BuiltInTok{id}\OperatorTok{=}\StringTok{"Acuna-Hidalgo-2017-5.8"}\OperatorTok{><}\NormalTok{span }\KeywordTok{class}\OperatorTok{=}\StringTok{"toc-item-num"}\OperatorTok{>}\FloatTok{5.8}\OperatorTok{&}\NormalTok{nbsp}\OperatorTok{;&}\NormalTok{nbsp}\OperatorTok{;</}\NormalTok{span}\OperatorTok{>}\NormalTok{Acuna}\OperatorTok{-}\NormalTok{Hidalgo }\DecValTok{2017}\OperatorTok{</}\NormalTok{a}\OperatorTok{></}\NormalTok{span}\OperatorTok{></}\NormalTok{li}\OperatorTok{></}\NormalTok{ul}\OperatorTok{></}\NormalTok{li}\OperatorTok{>}
    \OperatorTok{<}\NormalTok{li}\OperatorTok{><}\NormalTok{span}\OperatorTok{><}\NormalTok{a href}\OperatorTok{=}\StringTok{"#Lists-of-all-possible-variants-in-DNMT3A,-TET2,-ASXL1,-TP53"}\NormalTok{ data}\OperatorTok{-}\NormalTok{toc}\OperatorTok{-}\NormalTok{modified}\OperatorTok{-}\BuiltInTok{id}\OperatorTok{=}\StringTok{"Lists-of-all-possible-variants-in-DNMT3A,-TET2,-ASXL1,-TP53-6"}\OperatorTok{><}\NormalTok{span }\KeywordTok{class}\OperatorTok{=}\StringTok{"toc-item-num"}\OperatorTok{>}\DecValTok{6}\OperatorTok{&}\NormalTok{nbsp}\OperatorTok{;&}\NormalTok{nbsp}\OperatorTok{;</}\NormalTok{span}\OperatorTok{>}\NormalTok{Lists of }\BuiltInTok{all}\NormalTok{ possible variants }\KeywordTok{in}\NormalTok{ DNMT3A, TET2, ASXL1, TP53}\OperatorTok{</}\NormalTok{a}\OperatorTok{></}\NormalTok{span}\OperatorTok{>}
        \OperatorTok{<}\NormalTok{ul }\KeywordTok{class}\OperatorTok{=}\StringTok{"toc-item"}\OperatorTok{>}
        \OperatorTok{<}\NormalTok{li}\OperatorTok{><}\NormalTok{span}\OperatorTok{><}\NormalTok{a href}\OperatorTok{=}\StringTok{"#DNMT3A"}\NormalTok{ data}\OperatorTok{-}\NormalTok{toc}\OperatorTok{-}\NormalTok{modified}\OperatorTok{-}\BuiltInTok{id}\OperatorTok{=}\StringTok{"DNMT3A-6.1"}\OperatorTok{><}\NormalTok{span }\KeywordTok{class}\OperatorTok{=}\StringTok{"toc-item-num"}\OperatorTok{>}\FloatTok{6.1}\OperatorTok{&}\NormalTok{nbsp}\OperatorTok{;&}\NormalTok{nbsp}\OperatorTok{;</}\NormalTok{span}\OperatorTok{>}\NormalTok{DNMT3A}\OperatorTok{</}\NormalTok{a}\OperatorTok{></}\NormalTok{span}\OperatorTok{></}\NormalTok{li}\OperatorTok{>}
        \OperatorTok{<}\NormalTok{li}\OperatorTok{><}\NormalTok{span}\OperatorTok{><}\NormalTok{a href}\OperatorTok{=}\StringTok{"#TET2"}\NormalTok{ data}\OperatorTok{-}\NormalTok{toc}\OperatorTok{-}\NormalTok{modified}\OperatorTok{-}\BuiltInTok{id}\OperatorTok{=}\StringTok{"TET2-6.2"}\OperatorTok{><}\NormalTok{span }\KeywordTok{class}\OperatorTok{=}\StringTok{"toc-item-num"}\OperatorTok{>}\FloatTok{6.2}\OperatorTok{&}\NormalTok{nbsp}\OperatorTok{;&}\NormalTok{nbsp}\OperatorTok{;</}\NormalTok{span}\OperatorTok{>}\NormalTok{TET2}\OperatorTok{</}\NormalTok{a}\OperatorTok{></}\NormalTok{span}\OperatorTok{></}\NormalTok{li}\OperatorTok{>}
        \OperatorTok{<}\NormalTok{li}\OperatorTok{><}\NormalTok{span}\OperatorTok{><}\NormalTok{a href}\OperatorTok{=}\StringTok{"#ASXL1"}\NormalTok{ data}\OperatorTok{-}\NormalTok{toc}\OperatorTok{-}\NormalTok{modified}\OperatorTok{-}\BuiltInTok{id}\OperatorTok{=}\StringTok{"ASXL1-6.3"}\OperatorTok{><}\NormalTok{span }\KeywordTok{class}\OperatorTok{=}\StringTok{"toc-item-num"}\OperatorTok{>}\FloatTok{6.3}\OperatorTok{&}\NormalTok{nbsp}\OperatorTok{;&}\NormalTok{nbsp}\OperatorTok{;</}\NormalTok{span}\OperatorTok{>}\NormalTok{ASXL1}\OperatorTok{</}\NormalTok{a}\OperatorTok{></}\NormalTok{span}\OperatorTok{></}\NormalTok{li}\OperatorTok{>}
    \OperatorTok{<}\NormalTok{li}\OperatorTok{><}\NormalTok{span}\OperatorTok{><}\NormalTok{a href}\OperatorTok{=}\StringTok{"#TP53"}\NormalTok{ data}\OperatorTok{-}\NormalTok{toc}\OperatorTok{-}\NormalTok{modified}\OperatorTok{-}\BuiltInTok{id}\OperatorTok{=}\StringTok{"TP53-6.4"}\OperatorTok{><}\NormalTok{span }\KeywordTok{class}\OperatorTok{=}\StringTok{"toc-item-num"}\OperatorTok{>}\FloatTok{6.4}\OperatorTok{&}\NormalTok{nbsp}\OperatorTok{;&}\NormalTok{nbsp}\OperatorTok{;</}\NormalTok{span}\OperatorTok{>}\NormalTok{TP53}\OperatorTok{</}\NormalTok{a}\OperatorTok{></}\NormalTok{span}\OperatorTok{></}\NormalTok{li}\OperatorTok{></}\NormalTok{ul}\OperatorTok{></}\NormalTok{li}\OperatorTok{>}
    \OperatorTok{<}\NormalTok{li}\OperatorTok{><}\NormalTok{span}\OperatorTok{><}\NormalTok{a href}\OperatorTok{=}\StringTok{"#Actual-number-of-observations-of-each-variant"}\NormalTok{ data}\OperatorTok{-}\NormalTok{toc}\OperatorTok{-}\NormalTok{modified}\OperatorTok{-}\BuiltInTok{id}\OperatorTok{=}\StringTok{"Actual-number-of-observations-of-each-variant-7"}\OperatorTok{><}\NormalTok{span }\KeywordTok{class}\OperatorTok{=}\StringTok{"toc-item-num"}\OperatorTok{>}\DecValTok{7}\OperatorTok{&}\NormalTok{nbsp}\OperatorTok{;&}\NormalTok{nbsp}\OperatorTok{;</}\NormalTok{span}\OperatorTok{>}\NormalTok{Actual number of observations of each variant}\OperatorTok{</}\NormalTok{a}\OperatorTok{></}\NormalTok{span}\OperatorTok{>}
        \OperatorTok{<}\NormalTok{ul }\KeywordTok{class}\OperatorTok{=}\StringTok{"toc-item"}\OperatorTok{>}
        \OperatorTok{<}\NormalTok{li}\OperatorTok{>}
            \OperatorTok{<}\NormalTok{ul }\KeywordTok{class}\OperatorTok{=}\StringTok{"toc-item"}\OperatorTok{>}
            \OperatorTok{<}\NormalTok{li}\OperatorTok{><}\NormalTok{span}\OperatorTok{><}\NormalTok{a href}\OperatorTok{=}\StringTok{"#DNMT3A"}\NormalTok{ data}\OperatorTok{-}\NormalTok{toc}\OperatorTok{-}\NormalTok{modified}\OperatorTok{-}\BuiltInTok{id}\OperatorTok{=}\StringTok{"DNMT3A-7.0.1"}\OperatorTok{><}\NormalTok{span }\KeywordTok{class}\OperatorTok{=}\StringTok{"toc-item-num"}\OperatorTok{>}\FloatTok{7.0}\NormalTok{.}\DecValTok{1}\OperatorTok{&}\NormalTok{nbsp}\OperatorTok{;&}\NormalTok{nbsp}\OperatorTok{;</}\NormalTok{span}\OperatorTok{>}\NormalTok{DNMT3A}\OperatorTok{</}\NormalTok{a}\OperatorTok{></}\NormalTok{span}\OperatorTok{></}\NormalTok{li}\OperatorTok{>}
            \OperatorTok{<}\NormalTok{li}\OperatorTok{><}\NormalTok{span}\OperatorTok{><}\NormalTok{a href}\OperatorTok{=}\StringTok{"#TET2"}\NormalTok{ data}\OperatorTok{-}\NormalTok{toc}\OperatorTok{-}\NormalTok{modified}\OperatorTok{-}\BuiltInTok{id}\OperatorTok{=}\StringTok{"TET2-7.0.2"}\OperatorTok{><}\NormalTok{span }\KeywordTok{class}\OperatorTok{=}\StringTok{"toc-item-num"}\OperatorTok{>}\FloatTok{7.0}\NormalTok{.}\DecValTok{2}\OperatorTok{&}\NormalTok{nbsp}\OperatorTok{;&}\NormalTok{nbsp}\OperatorTok{;</}\NormalTok{span}\OperatorTok{>}\NormalTok{TET2}\OperatorTok{</}\NormalTok{a}\OperatorTok{></}\NormalTok{span}\OperatorTok{></}\NormalTok{li}\OperatorTok{>}
        \OperatorTok{<}\NormalTok{li}\OperatorTok{><}\NormalTok{span}\OperatorTok{><}\NormalTok{a href}\OperatorTok{=}\StringTok{"#ASXL1"}\NormalTok{ data}\OperatorTok{-}\NormalTok{toc}\OperatorTok{-}\NormalTok{modified}\OperatorTok{-}\BuiltInTok{id}\OperatorTok{=}\StringTok{"ASXL1-7.0.3"}\OperatorTok{><}\NormalTok{span }\KeywordTok{class}\OperatorTok{=}\StringTok{"toc-item-num"}\OperatorTok{>}\FloatTok{7.0}\NormalTok{.}\DecValTok{3}\OperatorTok{&}\NormalTok{nbsp}\OperatorTok{;&}\NormalTok{nbsp}\OperatorTok{;</}\NormalTok{span}\OperatorTok{>}\NormalTok{ASXL1}\OperatorTok{</}\NormalTok{a}\OperatorTok{></}\NormalTok{span}\OperatorTok{></}\NormalTok{li}\OperatorTok{>}
        \OperatorTok{<}\NormalTok{li}\OperatorTok{><}\NormalTok{span}\OperatorTok{><}\NormalTok{a href}\OperatorTok{=}\StringTok{"#TP53"}\NormalTok{ data}\OperatorTok{-}\NormalTok{toc}\OperatorTok{-}\NormalTok{modified}\OperatorTok{-}\BuiltInTok{id}\OperatorTok{=}\StringTok{"TP53-7.0.4"}\OperatorTok{><}\NormalTok{span }\KeywordTok{class}\OperatorTok{=}\StringTok{"toc-item-num"}\OperatorTok{>}\FloatTok{7.0}\NormalTok{.}\DecValTok{4}\OperatorTok{&}\NormalTok{nbsp}\OperatorTok{;&}\NormalTok{nbsp}\OperatorTok{;</}\NormalTok{span}\OperatorTok{>}\NormalTok{TP53}\OperatorTok{</}\NormalTok{a}\OperatorTok{></}\NormalTok{span}\OperatorTok{></}\NormalTok{li}\OperatorTok{></}\NormalTok{ul}\OperatorTok{></}\NormalTok{li}\OperatorTok{></}\NormalTok{ul}\OperatorTok{></}\NormalTok{li}\OperatorTok{>}
    \OperatorTok{<}\NormalTok{li}\OperatorTok{><}\NormalTok{span}\OperatorTok{><}\NormalTok{a href}\OperatorTok{=}\StringTok{"#Functions-for-calculating-the-expected-number-of-observations-of-a-variant"}\NormalTok{ data}\OperatorTok{-}\NormalTok{toc}\OperatorTok{-}\NormalTok{modified}\OperatorTok{-}\BuiltInTok{id}\OperatorTok{=}\StringTok{"Functions-for-calculating-the-expected-number-of-observations-of-a-variant-8"}\OperatorTok{><}\NormalTok{span }\KeywordTok{class}\OperatorTok{=}\StringTok{"toc-item-num"}\OperatorTok{>}\DecValTok{8}\OperatorTok{&}\NormalTok{nbsp}\OperatorTok{;&}\NormalTok{nbsp}\OperatorTok{;</}\NormalTok{span}\OperatorTok{>}\NormalTok{Functions }\ControlFlowTok{for}\NormalTok{ calculating the expected number of observations of a variant}\OperatorTok{</}\NormalTok{a}\OperatorTok{></}\NormalTok{span}\OperatorTok{></}\NormalTok{li}\OperatorTok{>}
    \OperatorTok{<}\NormalTok{li}\OperatorTok{><}\NormalTok{span}\OperatorTok{><}\NormalTok{a href}\OperatorTok{=}\StringTok{"#Maximum-Likelihood-Estimation-for-s"}\NormalTok{ data}\OperatorTok{-}\NormalTok{toc}\OperatorTok{-}\NormalTok{modified}\OperatorTok{-}\BuiltInTok{id}\OperatorTok{=}\StringTok{"Maximum-Likelihood-Estimation-for-s-9"}\OperatorTok{><}\NormalTok{span }\KeywordTok{class}\OperatorTok{=}\StringTok{"toc-item-num"}\OperatorTok{>}\DecValTok{9}\OperatorTok{&}\NormalTok{nbsp}\OperatorTok{;&}\NormalTok{nbsp}\OperatorTok{;</}\NormalTok{span}\OperatorTok{>}\NormalTok{Maximum Likelihood Estimation }\ControlFlowTok{for}\NormalTok{ s}\OperatorTok{</}\NormalTok{a}\OperatorTok{></}\NormalTok{span}\OperatorTok{>}
        \OperatorTok{<}\NormalTok{ul }\KeywordTok{class}\OperatorTok{=}\StringTok{"toc-item"}\OperatorTok{><}\NormalTok{li}\OperatorTok{><}\NormalTok{span}\OperatorTok{><}\NormalTok{a href}\OperatorTok{=}\StringTok{"#DNMT3A-variants"}\NormalTok{ data}\OperatorTok{-}\NormalTok{toc}\OperatorTok{-}\NormalTok{modified}\OperatorTok{-}\BuiltInTok{id}\OperatorTok{=}\StringTok{"DNMT3A-variants-9.1"}\OperatorTok{><}\NormalTok{span }\KeywordTok{class}\OperatorTok{=}\StringTok{"toc-item-num"}\OperatorTok{>}\FloatTok{9.1}\OperatorTok{&}\NormalTok{nbsp}\OperatorTok{;&}\NormalTok{nbsp}\OperatorTok{;</}\NormalTok{span}\OperatorTok{>}\NormalTok{DNMT3A variants}\OperatorTok{</}\NormalTok{a}\OperatorTok{></}\NormalTok{span}\OperatorTok{></}\NormalTok{li}\OperatorTok{>}
        \OperatorTok{<}\NormalTok{li}\OperatorTok{><}\NormalTok{span}\OperatorTok{><}\NormalTok{a href}\OperatorTok{=}\StringTok{"#TET2-variants"}\NormalTok{ data}\OperatorTok{-}\NormalTok{toc}\OperatorTok{-}\NormalTok{modified}\OperatorTok{-}\BuiltInTok{id}\OperatorTok{=}\StringTok{"TET2-variants-9.2"}\OperatorTok{><}\NormalTok{span }\KeywordTok{class}\OperatorTok{=}\StringTok{"toc-item-num"}\OperatorTok{>}\FloatTok{9.2}\OperatorTok{&}\NormalTok{nbsp}\OperatorTok{;&}\NormalTok{nbsp}\OperatorTok{;</}\NormalTok{span}\OperatorTok{>}\NormalTok{TET2 variants}\OperatorTok{</}\NormalTok{a}\OperatorTok{></}\NormalTok{span}\OperatorTok{></}\NormalTok{li}\OperatorTok{>}
        \OperatorTok{<}\NormalTok{li}\OperatorTok{><}\NormalTok{span}\OperatorTok{><}\NormalTok{a href}\OperatorTok{=}\StringTok{"#ASXL1-variants"}\NormalTok{ data}\OperatorTok{-}\NormalTok{toc}\OperatorTok{-}\NormalTok{modified}\OperatorTok{-}\BuiltInTok{id}\OperatorTok{=}\StringTok{"ASXL1-variants-9.3"}\OperatorTok{><}\NormalTok{span }\KeywordTok{class}\OperatorTok{=}\StringTok{"toc-item-num"}\OperatorTok{>}\FloatTok{9.3}\OperatorTok{&}\NormalTok{nbsp}\OperatorTok{;&}\NormalTok{nbsp}\OperatorTok{;</}\NormalTok{span}\OperatorTok{>}\NormalTok{ASXL1 variants}\OperatorTok{</}\NormalTok{a}\OperatorTok{></}\NormalTok{span}\OperatorTok{></}\NormalTok{li}\OperatorTok{>}
        \OperatorTok{<}\NormalTok{li}\OperatorTok{><}\NormalTok{span}\OperatorTok{><}\NormalTok{a href}\OperatorTok{=}\StringTok{"#TP53-variants"}\NormalTok{ data}\OperatorTok{-}\NormalTok{toc}\OperatorTok{-}\NormalTok{modified}\OperatorTok{-}\BuiltInTok{id}\OperatorTok{=}\StringTok{"TP53-variants-9.4"}\OperatorTok{><}\NormalTok{span }\KeywordTok{class}\OperatorTok{=}\StringTok{"toc-item-num"}\OperatorTok{>}\FloatTok{9.4}\OperatorTok{&}\NormalTok{nbsp}\OperatorTok{;&}\NormalTok{nbsp}\OperatorTok{;</}\NormalTok{span}\OperatorTok{>}\NormalTok{TP53 variants}\OperatorTok{</}\NormalTok{a}\OperatorTok{></}\NormalTok{span}\OperatorTok{></}\NormalTok{li}\OperatorTok{>}
        \OperatorTok{</}\NormalTok{ul}\OperatorTok{>}
        \OperatorTok{</}\NormalTok{li}\OperatorTok{>}
    \OperatorTok{</}\NormalTok{ul}\OperatorTok{>}
\OperatorTok{</}\NormalTok{div}\OperatorTok{>}
\end{Highlighting}
\end{Shaded}

\chapter{Visualization}\label{visualization}

\section{Color}\label{color}

\subsection{Colorschemes}\label{colorschemes}

Seaborn Themes

\begin{Shaded}
\begin{Highlighting}[]
\NormalTok{Pastel: \{}\StringTok{'Blue'}\NormalTok{:}\StringTok{'#a3c6ff'}\NormalTok{, }\StringTok{'Orange'}\NormalTok{:}\StringTok{'#f7ab60'}\NormalTok{, }\StringTok{'Green'}\NormalTok{:}\StringTok{'#60f7a9'}\NormalTok{, }\StringTok{'Red'}\NormalTok{:}\StringTok{'#fc9d94'}\NormalTok{, }\StringTok{'Purple'}\NormalTok{:}\StringTok{'#bea3ff'}\NormalTok{, }\StringTok{'Brown'}\NormalTok{:}\StringTok{'#d1b485'}\NormalTok{, }\StringTok{'Pink'}\NormalTok{:}\StringTok{'#f7afdf'}\NormalTok{, }\StringTok{'Gray'}\NormalTok{:}\StringTok{'#c4c4c4'}\NormalTok{, }\StringTok{'Yellow'}\NormalTok{:}\StringTok{'#ffffaa'}\NormalTok{, }\StringTok{'LBlue'}\NormalTok{:}\StringTok{'#baf6ff'}\NormalTok{\}}
\end{Highlighting}
\end{Shaded}

\begin{Shaded}
\begin{Highlighting}[]
\NormalTok{Deep: \{}\StringTok{'Green'}\NormalTok{:}\StringTok{'#5baf68'}\NormalTok{\}}
\end{Highlighting}
\end{Shaded}

\subsection{Controlling Coloration}\label{controlling-coloration}

Not all plots automatically plot with a white background, and when using
something dark like jupyterlab or a presentation this can be
frustrating. The background color can be set in pyplot like this.

\begin{Shaded}
\begin{Highlighting}[]
\NormalTok{fig.patch.set_facecolor(}\StringTok{'xkcd:mint green'}\NormalTok{)}
\end{Highlighting}
\end{Shaded}

When plotting, samples will not always be colored with the same color,
especially when different subsets of samples are included in different
plots. Here is a manual workaround to specify the coloration of
displayed data. This is a bit cumbersome so there might be a more
elegant way of achieving the same outcome.

\begin{Shaded}
\begin{Highlighting}[]
\CommentTok{# here is an example where sample order is controlled from a pandas DataFrame}
\NormalTok{sample_order }\OperatorTok{=}\NormalTok{ all_vars.sort_values([}\StringTok{'ID'}\NormalTok{]).drop_duplicates([}\StringTok{'Sample'}\NormalTok{]).Sample}

\CommentTok{# the color order is specified here}
\CommentTok{# colors should be in the same order as the above sample_order Series, excluding samples with no data}
\NormalTok{colors }\OperatorTok{=}\NormalTok{ [pastel[}\StringTok{'Brown'}\NormalTok{], pastel[}\StringTok{'Blue'}\NormalTok{],}
\NormalTok{          pastel[}\StringTok{'Orange'}\NormalTok{], pastel[}\StringTok{'Purple'}\NormalTok{],}
\NormalTok{          pastel[}\StringTok{'Green'}\NormalTok{], pastel[}\StringTok{'Red'}\NormalTok{],}
\NormalTok{          ]}

\NormalTok{plt.figure()}
\CommentTok{# this is an example of plotting that uses the sample_order and palette to control coloration order}
\NormalTok{sns.catplot(x}\OperatorTok{=}\StringTok{'Sample'}\NormalTok{, y}\OperatorTok{=}\StringTok{'VAF'}\NormalTok{, hue}\OperatorTok{=}\StringTok{'Gene'}\NormalTok{, jitter}\OperatorTok{=}\VariableTok{True}\NormalTok{,}
\NormalTok{            data}\OperatorTok{=}\NormalTok{oncogenic[oncogenic.Location }\OperatorTok{==} \StringTok{'Peripheral'}\NormalTok{],}
\NormalTok{            legend}\OperatorTok{=}\VariableTok{False}\NormalTok{, order}\OperatorTok{=}\NormalTok{sample_order, palette}\OperatorTok{=}\NormalTok{sns.color_palette(colors))}

\CommentTok{# a colorscheme can be specified if desired}
\NormalTok{pastel }\OperatorTok{=}\NormalTok{ \{}\StringTok{'Blue'}\NormalTok{:}\StringTok{'#a3c6ff'}\NormalTok{, }\StringTok{'Orange'}\NormalTok{:}\StringTok{'#f7ab60'}\NormalTok{,}
          \StringTok{'Green'}\NormalTok{:}\StringTok{'#60f7a9'}\NormalTok{, }\StringTok{'Red'}\NormalTok{:}\StringTok{'#fc9d94'}\NormalTok{,}
          \StringTok{'Purple'}\NormalTok{:}\StringTok{'#bea3ff'}\NormalTok{, }\StringTok{'Brown'}\NormalTok{:}\StringTok{'#d1b485'}\NormalTok{,}
          \StringTok{'Pink'}\NormalTok{:}\StringTok{'#f7afdf'}\NormalTok{, }\StringTok{'Gray'}\NormalTok{:}\StringTok{'#c4c4c4'}\NormalTok{,}
          \StringTok{'Yellow'}\NormalTok{:}\StringTok{'#ffffaa'}\NormalTok{, }\StringTok{'LBlue'}\NormalTok{:}\StringTok{'#baf6ff'}\NormalTok{\}}

\CommentTok{# this controls the coloration in the legend}
\ImportTok{import}\NormalTok{ matplotlib.patches }\ImportTok{as}\NormalTok{ mpatches}
\NormalTok{egfr }\OperatorTok{=}\NormalTok{ mpatches.Patch(color}\OperatorTok{=}\NormalTok{pastel[}\StringTok{'Blue'}\NormalTok{], label}\OperatorTok{=}\StringTok{'EGFR'}\NormalTok{)}
\NormalTok{pik3ca }\OperatorTok{=}\NormalTok{ mpatches.Patch(color}\OperatorTok{=}\NormalTok{pastel[}\StringTok{'Orange'}\NormalTok{], label}\OperatorTok{=}\StringTok{'PIK3CA'}\NormalTok{)}
\NormalTok{myc }\OperatorTok{=}\NormalTok{ mpatches.Patch(color}\OperatorTok{=}\NormalTok{pastel[}\StringTok{'Green'}\NormalTok{], label}\OperatorTok{=}\StringTok{'MYC'}\NormalTok{)}

\NormalTok{plt.legend(handles}\OperatorTok{=}\NormalTok{[egfr,pik3ca,myc],}
\NormalTok{           loc}\OperatorTok{=}\StringTok{'upper right'}\NormalTok{, bbox_to_anchor}\OperatorTok{=}\NormalTok{(}\FloatTok{1.5}\NormalTok{, }\DecValTok{1}\NormalTok{),}
\NormalTok{           ncol}\OperatorTok{=}\DecValTok{1}\NormalTok{) }\CommentTok{# no legend overlap}
\end{Highlighting}
\end{Shaded}

\section{Matplotlib}\label{matplotlib}

Plotting a heatmap.

\begin{Shaded}
\begin{Highlighting}[]
\ImportTok{import}\NormalTok{ matplotlib.pyplot }\ImportTok{as}\NormalTok{ plt}
\ImportTok{import}\NormalTok{ numpy }\ImportTok{as}\NormalTok{ np}
\NormalTok{a }\OperatorTok{=}\NormalTok{ np.random.random((}\DecValTok{16}\NormalTok{, }\DecValTok{16}\NormalTok{))}
\NormalTok{plt.imshow(a, cmap}\OperatorTok{=}\StringTok{'RdBu'', interpolation='}\NormalTok{nearest}\StringTok{')}
\StringTok{plt.show()}
\end{Highlighting}
\end{Shaded}

Possible heatmap colors are:

\begin{Shaded}
\begin{Highlighting}[]
\NormalTok{Accent, Accent_r, Blues, Blues_r, BrBG, BrBG_r, BuGn, BuGn_r, BuPu, BuPu_r, CMRmap, CMRmap_r, Dark2, Dark2_r, GnBu, GnBu_r, Greens, Greens_r, Greys, Greys_r, OrRd, OrRd_r, Oranges, Oranges_r, PRGn, PRGn_r, Paired, Paired_r, Pastel1, Pastel1_r, Pastel2, Pastel2_r, PiYG, PiYG_r, PuBu, PuBuGn, PuBuGn_r, PuBu_r, PuOr, PuOr_r, PuRd, PuRd_r, Purples, Purples_r, RdBu, RdBu_r, RdGy, RdGy_r, RdPu, RdPu_r, RdYlBu, RdYlBu_r, RdYlGn, RdYlGn_r, Reds, Reds_r, Set1,}
\NormalTok{Set1_r, Set2, Set2_r, Set3, Set3_r, Spectral, Spectral_r, Wistia, Wistia_r, YlGn, YlGnBu, YlGnBu_r, YlGn_r, YlOrBr, YlOrBr_r, YlOrRd, YlOrRd_r, afmhot, afmhot_r, autumn, autumn_r, binary, binary_r, bone, bone_r, brg, brg_r, bwr, bwr_r, cividis, cividis_r, cool, cool_r, coolwarm, coolwarm_r, copper, copper_r, cubehelix, cubehelix_r, flag, flag_r, gist_earth, gist_earth_r, gist_gray, gist_gray_r, gist_heat, gist_heat_r, gist_ncar, gist_ncar_r, gist_rainbow, gist_rainbow_r,}
\NormalTok{gist_stern, gist_stern_r, gist_yarg, gist_yarg_r, gnuplot, gnuplot2, gnuplot2_r, gnuplot_r, gray, gray_r, hot, hot_r, hsv, hsv_r, icefire, icefire_r, inferno, inferno_r, jet, jet_r, magma, magma_r, mako, mako_r, nipy_spectral, nipy_spectral_r, ocean, ocean_r, pink, pink_r, plasma, plasma_r, prism, prism_r, rainbow, rainbow_r, rocket, rocket_r, seismic, seismic_r, spring, spring_r, summer, summer_r, tab10, tab10_r, tab20, tab20_r, tab20b, tab20b_r, tab20c, tab20c_r, terrain, terrain_r,}
\NormalTok{twilight, twilight_r, twilight_shifted, twilight_shifted_r, viridis, viridis_r, vlag, vlag_r, winter, winter_r}
\end{Highlighting}
\end{Shaded}

A simple venn diagram.

\begin{Shaded}
\begin{Highlighting}[]
\ImportTok{from}\NormalTok{ matplotlib_venn }\ImportTok{import}\NormalTok{ venn2}
\NormalTok{venn2(subsets }\OperatorTok{=}\NormalTok{ (}\DecValTok{3}\NormalTok{, }\DecValTok{2}\NormalTok{, }\DecValTok{1}\NormalTok{))}
\end{Highlighting}
\end{Shaded}

A more complicated venn diagram.

\begin{Shaded}
\begin{Highlighting}[]
\ImportTok{from}\NormalTok{ matplotlib }\ImportTok{import}\NormalTok{ pyplot }\ImportTok{as}\NormalTok{ plt}
\ImportTok{import}\NormalTok{ numpy }\ImportTok{as}\NormalTok{ np}
\ImportTok{from}\NormalTok{ matplotlib_venn }\ImportTok{import}\NormalTok{ venn3, venn3_circles}
\NormalTok{plt.figure(figsize}\OperatorTok{=}\NormalTok{(}\DecValTok{4}\NormalTok{,}\DecValTok{4}\NormalTok{))}
\NormalTok{v }\OperatorTok{=}\NormalTok{ venn3(subsets}\OperatorTok{=}\NormalTok{(}\DecValTok{1}\NormalTok{, }\DecValTok{1}\NormalTok{, }\DecValTok{1}\NormalTok{, }\DecValTok{1}\NormalTok{, }\DecValTok{1}\NormalTok{, }\DecValTok{1}\NormalTok{, }\DecValTok{1}\NormalTok{), set_labels }\OperatorTok{=}\NormalTok{ (}\StringTok{'A'}\NormalTok{, }\StringTok{'B'}\NormalTok{, }\StringTok{'C'}\NormalTok{))}
\NormalTok{v.get_patch_by_id(}\StringTok{'100'}\NormalTok{).set_alpha(}\FloatTok{1.0}\NormalTok{)}
\NormalTok{v.get_patch_by_id(}\StringTok{'100'}\NormalTok{).set_color(}\StringTok{'white'}\NormalTok{)}
\NormalTok{v.get_label_by_id(}\StringTok{'100'}\NormalTok{).set_text(}\StringTok{'Unknown'}\NormalTok{)}
\NormalTok{v.get_label_by_id(}\StringTok{'A'}\NormalTok{).set_text(}\StringTok{'Set "A"'}\NormalTok{)}
\NormalTok{c }\OperatorTok{=}\NormalTok{ venn3_circles(subsets}\OperatorTok{=}\NormalTok{(}\DecValTok{1}\NormalTok{, }\DecValTok{1}\NormalTok{, }\DecValTok{1}\NormalTok{, }\DecValTok{1}\NormalTok{, }\DecValTok{1}\NormalTok{, }\DecValTok{1}\NormalTok{, }\DecValTok{1}\NormalTok{), linestyle}\OperatorTok{=}\StringTok{'dotted'}\NormalTok{)}
\NormalTok{c[}\DecValTok{0}\NormalTok{].set_lw(}\FloatTok{1.0}\NormalTok{)}
\NormalTok{c[}\DecValTok{0}\NormalTok{].set_ls(}\StringTok{'dotted'}\NormalTok{)}
\NormalTok{plt.title(}\StringTok{"Sample Venn diagram"}\NormalTok{)}
\NormalTok{plt.annotate(}\StringTok{'Unknown set'}\NormalTok{, xy}\OperatorTok{=}\NormalTok{v.get_label_by_id(}\StringTok{'100'}\NormalTok{).get_position() }\OperatorTok{-}\NormalTok{ np.array([}\DecValTok{0}\NormalTok{, }\FloatTok{0.05}\NormalTok{]), xytext}\OperatorTok{=}\NormalTok{(}\OperatorTok{-}\DecValTok{70}\NormalTok{,}\OperatorTok{-}\DecValTok{70}\NormalTok{),}
\NormalTok{             ha}\OperatorTok{=}\StringTok{'center'}\NormalTok{, textcoords}\OperatorTok{=}\StringTok{'offset points'}\NormalTok{, bbox}\OperatorTok{=}\BuiltInTok{dict}\NormalTok{(boxstyle}\OperatorTok{=}\StringTok{'round,pad=0.5'}\NormalTok{, fc}\OperatorTok{=}\StringTok{'gray'}\NormalTok{, alpha}\OperatorTok{=}\FloatTok{0.1}\NormalTok{),}
\NormalTok{                          arrowprops}\OperatorTok{=}\BuiltInTok{dict}\NormalTok{(arrowstyle}\OperatorTok{=}\StringTok{'->'}\NormalTok{, connectionstyle}\OperatorTok{=}\StringTok{'arc3,rad=0.5'}\NormalTok{,color}\OperatorTok{=}\StringTok{'gray'}\NormalTok{))}
\NormalTok{                          plt.show()}
\end{Highlighting}
\end{Shaded}

Log scales seem to always be a challenge. Here is at least one solution
to change ticks to log manually.

\begin{Shaded}
\begin{Highlighting}[]
\NormalTok{y_major_ticks }\OperatorTok{=}\NormalTok{ [np.log(}\DecValTok{100}\NormalTok{),np.log(}\DecValTok{200}\NormalTok{),np.log(}\DecValTok{300}\NormalTok{),np.log(}\DecValTok{400}\NormalTok{),np.log(}\DecValTok{500}\NormalTok{),np.log(}\DecValTok{600}\NormalTok{),np.log(}\DecValTok{700}\NormalTok{),np.log(}\DecValTok{800}\NormalTok{),np.log(}\DecValTok{900}\NormalTok{),}\OperatorTok{\textbackslash{}}
\NormalTok{                 np.log(}\DecValTok{1000}\NormalTok{),np.log(}\DecValTok{2000}\NormalTok{),np.log(}\DecValTok{3000}\NormalTok{),np.log(}\DecValTok{4000}\NormalTok{),np.log(}\DecValTok{5000}\NormalTok{),np.log(}\DecValTok{6000}\NormalTok{),np.log(}\DecValTok{7000}\NormalTok{),np.log(}\DecValTok{8000}\NormalTok{),np.log(}\DecValTok{9000}\NormalTok{),}\OperatorTok{\textbackslash{}}
\NormalTok{                 np.log(}\DecValTok{10000}\NormalTok{),np.log(}\DecValTok{20000}\NormalTok{),np.log(}\DecValTok{30000}\NormalTok{),np.log(}\DecValTok{40000}\NormalTok{),np.log(}\DecValTok{50000}\NormalTok{),np.log(}\DecValTok{60000}\NormalTok{),np.log(}\DecValTok{70000}\NormalTok{),np.log(}\DecValTok{80000}\NormalTok{),np.log(}\DecValTok{90000}\NormalTok{),}\OperatorTok{\textbackslash{}}
\NormalTok{                 np.log(}\DecValTok{100000}\NormalTok{),np.log(}\DecValTok{200000}\NormalTok{),np.log(}\DecValTok{300000}\NormalTok{),np.log(}\DecValTok{400000}\NormalTok{),np.log(}\DecValTok{500000}\NormalTok{),np.log(}\DecValTok{600000}\NormalTok{),np.log(}\DecValTok{700000}\NormalTok{),np.log(}\DecValTok{800000}\NormalTok{),np.log(}\DecValTok{900000}\NormalTok{),}\OperatorTok{\textbackslash{}}
\NormalTok{                 np.log(}\DecValTok{1000000}\NormalTok{),np.log(}\DecValTok{2000000}\NormalTok{),np.log(}\DecValTok{3000000}\NormalTok{),np.log(}\DecValTok{4000000}\NormalTok{),np.log(}\DecValTok{5000000}\NormalTok{),np.log(}\DecValTok{6000000}\NormalTok{),np.log(}\DecValTok{7000000}\NormalTok{),np.log(}\DecValTok{8000000}\NormalTok{),np.log(}\DecValTok{9000000}\NormalTok{),}\OperatorTok{\textbackslash{}}
\NormalTok{                 np.log(}\DecValTok{10000000}\NormalTok{)]}

\NormalTok{y_major_tick_labels }\OperatorTok{=}\NormalTok{ [}\StringTok{"100"}\NormalTok{,}\StringTok{""}\NormalTok{,}\StringTok{""}\NormalTok{,}\StringTok{""}\NormalTok{,}\StringTok{""}\NormalTok{,}\StringTok{""}\NormalTok{,}\StringTok{""}\NormalTok{,}\StringTok{""}\NormalTok{,}\StringTok{""}\NormalTok{, }\StringTok{"1000"}\NormalTok{,}\StringTok{""}\NormalTok{,}\StringTok{""}\NormalTok{,}\StringTok{""}\NormalTok{,}\StringTok{""}\NormalTok{,}\StringTok{""}\NormalTok{,}\StringTok{""}\NormalTok{,}\StringTok{""}\NormalTok{,}\StringTok{""}\NormalTok{, }\StringTok{"10,000"}\NormalTok{,}\OperatorTok{\textbackslash{}}
                       \StringTok{""}\NormalTok{,}\StringTok{""}\NormalTok{,}\StringTok{""}\NormalTok{,}\StringTok{""}\NormalTok{,}\StringTok{""}\NormalTok{,}\StringTok{""}\NormalTok{,}\StringTok{""}\NormalTok{,}\StringTok{""}\NormalTok{,}\StringTok{"100,000"}\NormalTok{,}\StringTok{""}\NormalTok{,}\StringTok{""}\NormalTok{,}\StringTok{""}\NormalTok{,}\StringTok{""}\NormalTok{,}\StringTok{""}\NormalTok{,}\StringTok{""}\NormalTok{,}\StringTok{""}\NormalTok{,}\StringTok{""}\NormalTok{, }\StringTok{"1,000,000"}\NormalTok{,}\StringTok{""}\NormalTok{,}\StringTok{""}\NormalTok{,}\StringTok{""}\NormalTok{,}\StringTok{""}\NormalTok{,}\StringTok{""}\NormalTok{,}\StringTok{""}\NormalTok{,}\StringTok{""}\NormalTok{,}\StringTok{""}\NormalTok{, }\StringTok{"10,000,000"}\NormalTok{ ]}
\NormalTok{ax1.set_yticks(y_major_ticks)}
\NormalTok{ax1.set_yticklabels(y_major_tick_labels, fontsize }\OperatorTok{=}\NormalTok{ axisfont)}
\NormalTok{ax1.yaxis.set_tick_params(width}\OperatorTok{=}\NormalTok{scale, color }\OperatorTok{=}\NormalTok{ grey3, length }\OperatorTok{=} \DecValTok{6}\NormalTok{)}
\end{Highlighting}
\end{Shaded}

\section{Seaborn}\label{seaborn}

Here is a general bar plot that includes some commonly used parameters.

\begin{Shaded}
\begin{Highlighting}[]
\CommentTok{# fits my 22 inch monitor}
\NormalTok{plt.figure(figsize}\OperatorTok{=}\NormalTok{(}\FloatTok{19.17}\NormalTok{,}\FloatTok{11.98}\NormalTok{))}
\CommentTok{# order controls the display order of the samples}
\NormalTok{sns.catplot(x}\OperatorTok{=}\StringTok{"Sample"}\NormalTok{, y}\OperatorTok{=}\StringTok{"Somatic"}\NormalTok{, kind}\OperatorTok{=}\StringTok{"bar"}\NormalTok{, data}\OperatorTok{=}\NormalTok{var_counts, order}\OperatorTok{=}\NormalTok{labels)}\OperatorTok{;}
\CommentTok{# keeps x-axis labels, but eliminates the tick mark}
\NormalTok{plt.tick_params(labelbottom}\OperatorTok{=}\VariableTok{True}\NormalTok{, bottom}\OperatorTok{=}\VariableTok{False}\NormalTok{)}
\CommentTok{# trim off the x-axis}
\NormalTok{sns.despine(offset}\OperatorTok{=}\DecValTok{10}\NormalTok{, trim}\OperatorTok{=}\VariableTok{True}\NormalTok{, bottom}\OperatorTok{=}\VariableTok{True}\NormalTok{)}
\CommentTok{# labels}
\NormalTok{plt.title(}\StringTok{''}\NormalTok{)}
\NormalTok{plt.ylabel(}\StringTok{''}\NormalTok{, fontsize}\OperatorTok{=}\DecValTok{8}\NormalTok{)}
\NormalTok{plt.xlabel(}\StringTok{''}\NormalTok{, fontsize}\OperatorTok{=}\DecValTok{8}\NormalTok{)}
\CommentTok{# manual control of xlabels}
\NormalTok{labels }\OperatorTok{=}\NormalTok{ [}\StringTok{'Indiv_1-a'}\NormalTok{,}\StringTok{'Indiv_2'}\NormalTok{,}\StringTok{'Indiv_3'}\NormalTok{,}\StringTok{'Indiv_1-b'}\NormalTok{]}
\CommentTok{# control xtick order}
\NormalTok{plt.xticks(}\BuiltInTok{range}\NormalTok{(}\BuiltInTok{len}\NormalTok{(labels)), labels, rotation}\OperatorTok{=}\DecValTok{45}\NormalTok{)}
\CommentTok{# control the number of x-ticks}
\NormalTok{plt.locator_params(axis}\OperatorTok{=}\StringTok{'x'}\NormalTok{, nbins}\OperatorTok{=}\DecValTok{10}\NormalTok{)}
\CommentTok{# legend positioning}
\NormalTok{plt.legend(loc}\OperatorTok{=}\StringTok{'upper right'}\NormalTok{)}
\CommentTok{# log scale}
\NormalTok{plt.gca().set_yscale(}\StringTok{'log'}\NormalTok{)}
\CommentTok{# this is better if neg values are needed}
\NormalTok{plt.gca().set_yscale(}\StringTok{'symlog'}\NormalTok{)}
\CommentTok{# fit plot to display}
\NormalTok{plt.tight_layout()}
\NormalTok{plt.show()}
\CommentTok{# save figure with tight_layout}
\NormalTok{plt.savefig(}\StringTok{"test.svg"}\NormalTok{, }\BuiltInTok{format}\OperatorTok{=}\StringTok{"svg"}\NormalTok{, bbox_inches}\OperatorTok{=}\StringTok{"tight"}\NormalTok{, dpi}\OperatorTok{=}\DecValTok{1000}\NormalTok{)}
\end{Highlighting}
\end{Shaded}

Signifance information can be added by including p-values and label bars
using the following code.

\begin{Shaded}
\begin{Highlighting}[]
\NormalTok{x1, x2 }\OperatorTok{=} \DecValTok{0}\NormalTok{, }\DecValTok{1} \CommentTok{# columns to annotate on the plot}
\NormalTok{y2, y1 }\OperatorTok{=} \DecValTok{20}\NormalTok{, }\DecValTok{15} \CommentTok{# placement of the line and how for down the vertical legs go}
\NormalTok{plt.plot([x1,x1, x2, x2], [y1, y2, y2, y1], linewidth}\OperatorTok{=}\DecValTok{1}\NormalTok{, color}\OperatorTok{=}\StringTok{'k'}\NormalTok{) }\CommentTok{# stats line}
\NormalTok{plt.text((x1}\OperatorTok{+}\NormalTok{x2)}\OperatorTok{*}\NormalTok{.}\DecValTok{5}\NormalTok{, y2}\OperatorTok{+}\DecValTok{2}\NormalTok{, }\StringTok{"p=0.09"}\NormalTok{, ha}\OperatorTok{=}\StringTok{'center'}\NormalTok{, va}\OperatorTok{=}\StringTok{'bottom'}\NormalTok{, fontsize}\OperatorTok{=}\DecValTok{8}\NormalTok{) }\CommentTok{# p-value or sig}
\end{Highlighting}
\end{Shaded}

\section{Statistics}\label{statistics}

This is a two-sided T-test for the null hypothesis that two populations
have the same means. It is important to note that it assumes the
population variances are the same, so this must be changed if the
assumption is incorrect.

\begin{Shaded}
\begin{Highlighting}[]
\CommentTok{# ttest_ind(a, b, axis=0, equal_var=True, nan_policy='propagate')}
\ImportTok{from}\NormalTok{ scipy.stats }\ImportTok{import}\NormalTok{ ttest_ind}
\NormalTok{ttest_ind(df[df[}\StringTok{'sample'}\NormalTok{] }\OperatorTok{==} \StringTok{'one'}\NormalTok{][}\StringTok{'means'}\NormalTok{], df[df[}\StringTok{'sample'}\NormalTok{] }\OperatorTok{==} \StringTok{'two'}\NormalTok{][}\StringTok{'means'}\NormalTok{])}
\end{Highlighting}
\end{Shaded}

\section{Various Plot Styles}\label{various-plot-styles}

This displays each individual datapoint overlayed on a boxplot

\begin{Shaded}
\begin{Highlighting}[]
\NormalTok{ax }\OperatorTok{=}\NormalTok{ sns.boxplot(x}\OperatorTok{=}\StringTok{'day'}\NormalTok{, y}\OperatorTok{=}\StringTok{'total_bill'}\NormalTok{, data}\OperatorTok{=}\NormalTok{tips)}
\NormalTok{ax }\OperatorTok{=}\NormalTok{ sns.swarmplot(x}\OperatorTok{=}\StringTok{'day'}\NormalTok{, y}\OperatorTok{=}\StringTok{'total_bill'}\NormalTok{, data}\OperatorTok{=}\NormalTok{tips, color}\OperatorTok{=}\StringTok{'.25'}\NormalTok{)}
\end{Highlighting}
\end{Shaded}

\chapter{Biology}\label{biology}

\section{General}\label{general}

Some helpful commands for genetic sequence.

\begin{Shaded}
\begin{Highlighting}[]
\ImportTok{from}\NormalTok{ string }\ImportTok{import}\NormalTok{ ascii_uppercase }\CommentTok{# python 3}
\ImportTok{from}\NormalTok{ string }\ImportTok{import}\NormalTok{ upper, lower }\CommentTok{# python 2}
\NormalTok{upper(}\StringTok{'tcga'}\NormalTok{)}
\NormalTok{lower(}\StringTok{'TCGA'}\NormalTok{)}
\NormalTok{title(}\StringTok{'tcga'}\NormalTok{) }\CommentTok{# capitalize the first letter}
\end{Highlighting}
\end{Shaded}

\section{Biopython}\label{biopython}

Reverse complement of sequence

\begin{Shaded}
\begin{Highlighting}[]
\ImportTok{from}\NormalTok{ Bio.Seq }\ImportTok{import}\NormalTok{ Seq}
\BuiltInTok{str}\NormalTok{(Seq(i).reverse_complement())}
\end{Highlighting}
\end{Shaded}

\section{UCSC Genome Browser}\label{ucsc-genome-browser}

Get sequence from UCSC genome browser

\begin{Shaded}
\begin{Highlighting}[]
\ImportTok{from}\NormalTok{ subprocess }\ImportTok{import}\NormalTok{ check_output, STDOUT}
\NormalTok{temp }\OperatorTok{=}\NormalTok{ check_output(}\StringTok{'wget -qO- http://genome.ucsc.edu/cgi-bin/das/hg19/dna?segment=}\SpecialCharTok{%s}\StringTok{:}\SpecialCharTok{%s}\StringTok{,}\SpecialCharTok\NormalTok{ (vcfObj.chrom,low,high), stderr}\OperatorTok{=}\NormalTok{STDOUT, shell}\OperatorTok{=}\VariableTok{True}\NormalTok{)}
\end{Highlighting}
\end{Shaded}

\section{Ref Genome}\label{ref-genome}

Get sequence from reference genome

\begin{Shaded}
\begin{Highlighting}[]
\ImportTok{from}\NormalTok{ subprocess }\ImportTok{import}\NormalTok{ check_output, STDOUT}
\NormalTok{temp }\OperatorTok{=}\NormalTok{ check_output(}\StringTok{'samtools faidx }\SpecialCharTok{%s}\StringTok{ }\SpecialCharTok{%s}\StringTok{:}\SpecialCharTok{%s}\StringTok{-}\SpecialCharTok\NormalTok{ (ref, vcfObj.chrom, low, high), stderr}\OperatorTok{=}\NormalTok{STDOUT, shell}\OperatorTok{=}\VariableTok{True}\NormalTok{)}

\NormalTok{finalSeq }\OperatorTok{=} \StringTok{''}
\ControlFlowTok{for}\NormalTok{ line }\KeywordTok{in}\NormalTok{ temp.decode(}\StringTok{'UTF-8'}\NormalTok{).split(}\StringTok{'}\CharTok{\textbackslash{}n}\StringTok{'}\NormalTok{):}
\ControlFlowTok{for}\NormalTok{ line }\KeywordTok{in}\NormalTok{ temp.decode(}\StringTok{'UTF-8'}\NormalTok{).split(}\StringTok{'}\CharTok{\textbackslash{}n}\StringTok{'}\NormalTok{): }\CommentTok{# this is only necessary in python 3 to convert binary to string}
    \ControlFlowTok{if} \StringTok{'>'} \KeywordTok{not} \KeywordTok{in}\NormalTok{ line:}
\NormalTok{        finalSeq }\OperatorTok{+=}\NormalTok{ line}

\NormalTok{finalSeq }\OperatorTok{=}\NormalTok{ finalSeq.upper()}
\end{Highlighting}
\end{Shaded}

\section{Personal Information}\label{personal-information}

\begin{Shaded}
\begin{Highlighting}[]
\CommentTok{# parse vcf file with parseline}
\ControlFlowTok{if} \StringTok{'#'} \KeywordTok{not} \KeywordTok{in}\NormalTok{ line }\KeywordTok{and} \StringTok{'chr'} \KeywordTok{in}\NormalTok{ line: }\CommentTok{# skip the info}
\CommentTok{# vcf handling}
\ImportTok{from}\NormalTok{ parseline }\ImportTok{import}\NormalTok{ VCFObj}
\CommentTok{# or}
\ImportTok{from}\NormalTok{ util }\ImportTok{import}\NormalTok{ VCFObj}
\NormalTok{vcfObj }\OperatorTok{=}\NormalTok{ VCFObj(vcfLine)}
\CommentTok{# available attributes: ao, dp, af, wt, var, chrom, location}
\end{Highlighting}
\end{Shaded}

\chapter{Data I/O}\label{io}

\section{Reading Data Files}\label{reading-data-files}

Opening .gz files

\begin{Shaded}
\begin{Highlighting}[]
\ImportTok{import}\NormalTok{ gzip}
\ControlFlowTok{for}\NormalTok{ line }\KeywordTok{in}\NormalTok{ gzip.}\BuiltInTok{open}\NormalTok{(}\StringTok{'myFile.gz'}\NormalTok{):}
    \BuiltInTok{print}\NormalTok{ line}
\end{Highlighting}
\end{Shaded}

\section{Pickles}\label{pickles}

Writing data in pickle format

\begin{Shaded}
\begin{Highlighting}[]
\ImportTok{import}\NormalTok{ pickle}
\NormalTok{p }\OperatorTok{=} \BuiltInTok{open}\NormalTok{(}\StringTok{'principle.pkl'}\NormalTok{, }\StringTok{'wb'}\NormalTok{)}
\NormalTok{pickle.dump(principleData, p)}
\NormalTok{p.close()}
\end{Highlighting}
\end{Shaded}

Reading data in pickle format

\begin{Shaded}
\begin{Highlighting}[]
\ImportTok{import}\NormalTok{ pickle}
\NormalTok{p }\OperatorTok{=} \BuiltInTok{open}\NormalTok{(}\StringTok{'principle.pkl'}\NormalTok{, }\StringTok{'rb'}\NormalTok{)}
\NormalTok{principleData }\OperatorTok{=}\NormalTok{ pickle.load(p)}
\NormalTok{p.close()}
\end{Highlighting}
\end{Shaded}

\chapter{Pandas}\label{pandas}

\section{File I/O}\label{file-io}

Read a csv file into a DataFrame.

\begin{Shaded}
\begin{Highlighting}[]
\NormalTok{pd.read_csv(filepath)}
\end{Highlighting}
\end{Shaded}

\section{Relabeling}\label{relabeling}

Rename a column or group of columns can be done by passing a dictionary
of the changes.

\begin{Shaded}
\begin{Highlighting}[]
\NormalTok{    df }\OperatorTok{=}\NormalTok{ df.rename(columns}\OperatorTok{=}\NormalTok{\{}\StringTok{'a'}\NormalTok{:}\StringTok{'b'}\NormalTok{,}\StringTok{'c'}\NormalTok{:}\StringTok{'d'}\NormalTok{\})}
\end{Highlighting}
\end{Shaded}

\section{Sorting and Arranging}\label{sorting-and-arranging}

The data in a DataFrame can be sorted in numeric or lexicographic order.
The following code sorts the values within the columns a and b.

\begin{Shaded}
\begin{Highlighting}[]
\NormalTok{df.sort_values([}\StringTok{'a'}\NormalTok{,}\StringTok{'b'}\NormalTok{])}
\end{Highlighting}
\end{Shaded}

\section{Editing Data}\label{editing-data}

Drop columns from a DataFrame.

\begin{Shaded}
\begin{Highlighting}[]
\ImportTok{import}\NormalTok{ numpy }\ImportTok{as}\NormalTok{ np}
\NormalTok{df }\OperatorTok{=}\NormalTok{ pd.DataFrame(np.arange(}\DecValTok{12}\NormalTok{).reshape(}\DecValTok{3}\NormalTok{,}\DecValTok{4}\NormalTok{),}
\NormalTok{                    columns}\OperatorTok{=}\NormalTok{[}\StringTok{'A'}\NormalTok{, }\StringTok{'B'}\NormalTok{, }\StringTok{'C'}\NormalTok{, }\StringTok{'D'}\NormalTok{])}
\BuiltInTok{print}\NormalTok{(df)}

\NormalTok{df }\OperatorTok{=}\NormalTok{ df.drop(columns}\OperatorTok{=}\NormalTok{[}\StringTok{'B'}\NormalTok{, }\StringTok{'C'}\NormalTok{]) }\CommentTok{# may not work in python 2}
\NormalTok{df }\OperatorTok{=}\NormalTok{ df.drop([}\StringTok{'B'}\NormalTok{, }\StringTok{'C'}\NormalTok{], axis}\OperatorTok{=}\DecValTok{1}\NormalTok{) }\CommentTok{# this works in python 2}
\BuiltInTok{print}\NormalTok{(df)}
\end{Highlighting}
\end{Shaded}

Changing the datatype of a column of data can be done by just changing
column type.

\begin{Shaded}
\begin{Highlighting}[]
\NormalTok{df.Age }\OperatorTok{=}\NormalTok{ df.Age.astype(}\BuiltInTok{str}\NormalTok{)}
\end{Highlighting}
\end{Shaded}

\subsection{Replace values}\label{replace-values}

New data can be set within a DataFrame one subset at a time in a way
that will avoid the SettingWithCopyWarning.

\begin{Shaded}
\begin{Highlighting}[]
\ImportTok{import}\NormalTok{ pandas }\ImportTok{as}\NormalTok{ pd}
\NormalTok{df }\OperatorTok{=}\NormalTok{ pd.DataFrame(\{}\StringTok{'Trait'}\NormalTok{:[}\StringTok{'Seed_Shape'}\NormalTok{,}\StringTok{'Seed_Shape'}\NormalTok{,}\StringTok{'Flower_Color'}\NormalTok{,}\StringTok{'Flower_Color'}\NormalTok{],}
                    \StringTok{'Phenotype'}\NormalTok{:[}\StringTok{'Round'}\NormalTok{,}\StringTok{'Wrinkled'}\NormalTok{,}\StringTok{'Purple'}\NormalTok{,}\StringTok{'White'}\NormalTok{]\})}
\NormalTok{df.loc[df.Trait }\OperatorTok{==} \StringTok{'Seed_Shape'}\NormalTok{, }\StringTok{'Affected_Part'}\NormalTok{] }\OperatorTok{=} \StringTok{'Seed'}
\NormalTok{df.loc[df.Trait }\OperatorTok{==} \StringTok{'Flower_Color'}\NormalTok{, }\StringTok{'Affected_Part'}\NormalTok{] }\OperatorTok{=} \StringTok{'Flower'}
\BuiltInTok{print}\NormalTok{(df)}
\end{Highlighting}
\end{Shaded}

\begin{verbatim}
##           Trait Phenotype Affected_Part
## 0    Seed_Shape     Round          Seed
## 1    Seed_Shape  Wrinkled          Seed
## 2  Flower_Color    Purple        Flower
## 3  Flower_Color     White        Flower
\end{verbatim}

There is a more simple alternative to the above method buit it may
result in the SettingWithCopyWarning.

\begin{Shaded}
\begin{Highlighting}[]
\NormalTok{df }\OperatorTok{=}\NormalTok{ df.replace(}\StringTok{'pork'}\NormalTok{,}\StringTok{'bacon'}\NormalTok{)}
\end{Highlighting}
\end{Shaded}

\section{Combining Data Structures}\label{combining-data-structures}

The following merges df and df2 using inner to get the intersection on
the Sample column, where indexes are ignored if the merging is performed
on a column as in the following example. The other possible merging
strategies are: left: use only keys from left frame, similar to a SQL
left outer join; preserve key order. right: use only keys from right
frame, similar to a SQL right outer join; preserve key order. outer: use
union of keys from both frames, similar to a SQL full outer join; sort
keys lexicographically. inner: use intersection of keys from both
frames, similar to a SQL inner join; preserve the order of the left
keys.

\begin{Shaded}
\begin{Highlighting}[]
\NormalTok{df }\OperatorTok{=}\NormalTok{ pd.merge(df, df2, how}\OperatorTok{=}\StringTok{'inner'}\NormalTok{, on}\OperatorTok{=}\NormalTok{[}\StringTok{'Sample'}\NormalTok{])}
\end{Highlighting}
\end{Shaded}

Appending to a Dataframe attaches a DataFrame after another one.

\begin{Shaded}
\begin{Highlighting}[]
\NormalTok{df }\OperatorTok{=}\NormalTok{ pd.DataFrame([[}\DecValTok{1}\NormalTok{, }\DecValTok{2}\NormalTok{], [}\DecValTok{3}\NormalTok{, }\DecValTok{4}\NormalTok{]], columns}\OperatorTok{=}\BuiltInTok{list}\NormalTok{(}\StringTok{'AB'}\NormalTok{))}
\NormalTok{df2 }\OperatorTok{=}\NormalTok{ pd.DataFrame([[}\DecValTok{5}\NormalTok{, }\DecValTok{6}\NormalTok{], [}\DecValTok{7}\NormalTok{, }\DecValTok{8}\NormalTok{]], columns}\OperatorTok{=}\BuiltInTok{list}\NormalTok{(}\StringTok{'AB'}\NormalTok{))}
\NormalTok{df.append(df2)}
\end{Highlighting}
\end{Shaded}

\section{Splitting}\label{splitting}

Remove duplicates

\begin{Shaded}
\begin{Highlighting}[]
\NormalTok{x }\OperatorTok{=}\NormalTok{ x[}\OperatorTok{~}\NormalTok{x.index.duplicated(keep}\OperatorTok{=}\StringTok{'first'}\NormalTok{)] }\CommentTok{# most ideal method}

\NormalTok{data }\OperatorTok{=}\NormalTok{ pd.DataFrame(\{}\StringTok{'k1'}\NormalTok{:[}\StringTok{'one'}\NormalTok{,}\StringTok{'two'}\NormalTok{]}\OperatorTok{*}\DecValTok{3}\OperatorTok{+}\NormalTok{[}\StringTok{'two'}\NormalTok{],}\StringTok{'k2'}\NormalTok{:[}\DecValTok{1}\NormalTok{,}\DecValTok{1}\NormalTok{,}\DecValTok{2}\NormalTok{,}\DecValTok{3}\NormalTok{,}\DecValTok{3}\NormalTok{,}\DecValTok{4}\NormalTok{,}\DecValTok{4}\NormalTok{]\})}
\NormalTok{data.duplicated() }\CommentTok{# identify duplicate data}
\NormalTok{data[‘k1’].duplicated()}
\NormalTok{data[}\StringTok{'k1'}\NormalTok{].drop_duplicates()}
\NormalTok{data.drop_duplicates[}\StringTok{'k1'}\NormalTok{] }\CommentTok{# this does the same thing as the previous line}
\NormalTok{data.drop_duplicates([}\StringTok{'k1'}\NormalTok{,}\StringTok{'k2'}\NormalTok{], keep}\OperatorTok{=}\StringTok{'last'}\NormalTok{) }\CommentTok{# drops unique found in k1 and k2 and keeps the last indexed duplicate}
\end{Highlighting}
\end{Shaded}

\section{Summarizing}\label{summarizing}

The mean of column values can be calculated where each of the columns is
grouped by the data in a specified column.

\begin{Shaded}
\begin{Highlighting}[]
\NormalTok{temp[[}\StringTok{'Sample'}\NormalTok{,}\StringTok{'VAF'}\NormalTok{,}\StringTok{'Var_Count'}\NormalTok{]].groupby(}\StringTok{'Sample'}\NormalTok{).mean()}
\end{Highlighting}
\end{Shaded}

\section{Arithmetic and Row-wise
Analysis}\label{arithmetic-and-row-wise-analysis}

Sometimes it is helpful to analyze the value in a particular cell in a
conditional manner depending on it's value and then set the result of
this analysis to a corresponding cell in a new column. Here is an
example where the VAF of a variant is conditionally analyzed

\begin{Shaded}
\begin{Highlighting}[]
\KeywordTok{def}\NormalTok{ LOH(x):}
    \ControlFlowTok{if}\NormalTok{ x }\OperatorTok{>} \FloatTok{0.75}\NormalTok{: }\ControlFlowTok{return} \DecValTok{1} \OperatorTok{-}\NormalTok{ x}
    \ControlFlowTok{elif}\NormalTok{ x }\OperatorTok{<=} \FloatTok{0.75} \KeywordTok{and}\NormalTok{ x }\OperatorTok{>} \FloatTok{0.25}\NormalTok{: }\ControlFlowTok{return} \BuiltInTok{abs}\NormalTok{(}\FloatTok{0.5} \OperatorTok{-}\NormalTok{ x)}
    \ControlFlowTok{else}\NormalTok{: }\ControlFlowTok{return} \DecValTok{0}
\NormalTok{all_vars[}\StringTok{'LOH'}\NormalTok{] }\OperatorTok{=}\NormalTok{ all_vars.VAF.transform(LOH)}
\NormalTok{max_loh }\OperatorTok{=}\NormalTok{ all_vars.groupby(}\StringTok{'Sample'}\NormalTok{).LOH.}\BuiltInTok{max}\NormalTok{().reset_index().rename(columns}\OperatorTok{=}\NormalTok{\{}\StringTok{'LOH'}\NormalTok{:}\StringTok{'Max_LOH'}\NormalTok{\})}
\NormalTok{all_vars }\OperatorTok{=}\NormalTok{ pd.merge(all_vars, max_loh, how}\OperatorTok{=}\StringTok{'inner'}\NormalTok{, on}\OperatorTok{=}\NormalTok{[}\StringTok{'Sample'}\NormalTok{])}
\end{Highlighting}
\end{Shaded}

\chapter{Git}\label{git}

\section{Setup}\label{setup}

\subsection{Git Setup}\label{git-setup}

The username and email needs to be added after git is installed.

\begin{Shaded}
\begin{Highlighting}[]
\FunctionTok{git}\NormalTok{ config --global user.name }\StringTok{"me"}
\FunctionTok{git}\NormalTok{ config --global user.email }\StringTok{"me@gmail.com"}
\end{Highlighting}
\end{Shaded}

After this information has been set, it can be checked.

\begin{Shaded}
\begin{Highlighting}[]
\FunctionTok{git}\NormalTok{ config --list}
\end{Highlighting}
\end{Shaded}

\subsection{Repository Initiation}\label{repository-initiation}

To setup a repository, create a folder with an initial file like a
README and then initiate it.

\begin{Shaded}
\begin{Highlighting}[]
\FunctionTok{git}\NormalTok{ init}
\FunctionTok{git}\NormalTok{ status}
\end{Highlighting}
\end{Shaded}

\subsection{Mirror on Online
Repository}\label{mirror-on-online-repository}

Create a repository on a repository like github, gitlab, bitbucket, or
sourceforge. Then the local git repository can be synched with the
online repository.

\begin{Shaded}
\begin{Highlighting}[]
\FunctionTok{git}\NormalTok{ remote add origin url-of-online-repository-here}
\FunctionTok{git}\NormalTok{ push -u origin master}
\end{Highlighting}
\end{Shaded}

Of course the repository could just be setup first and then cloned.

\begin{Shaded}
\begin{Highlighting}[]
\FunctionTok{git}\NormalTok{ clone url-of-online-repository-here}
\end{Highlighting}
\end{Shaded}

\section{Manipulating Commits}\label{manipulating-commits}

\subsection{Repository Status}\label{repository-status}

The commit history of a repository can be displayed in verbose form and
in summarized form.

\begin{Shaded}
\begin{Highlighting}[]
\FunctionTok{git}\NormalTok{ log}
\FunctionTok{git}\NormalTok{ log --oneline}
\end{Highlighting}
\end{Shaded}

\subsection{File Checkout}\label{file-checkout}

To restore a previous version of a file it can be checked out by first
identifying the version to be used using the log history and then
restoring the desired file.

\begin{Shaded}
\begin{Highlighting}[]
\FunctionTok{git}\NormalTok{ log --oneline}
\FunctionTok{git}\NormalTok{ checkout }\OperatorTok{<}\NormalTok{commit number}\OperatorTok{>}\NormalTok{ file.txt}
\end{Highlighting}
\end{Shaded}

\subsection{Resetting a Repository}\label{resetting-a-repository}

To discard the effect of the previous operation on a file.

\begin{Shaded}
\begin{Highlighting}[]
\FunctionTok{git}\NormalTok{ reset HEAD file.txt}
\end{Highlighting}
\end{Shaded}

The previous version of the a file can then be restored.

\begin{Shaded}
\begin{Highlighting}[]
\FunctionTok{git}\NormalTok{ checkout -- file.txt}
\end{Highlighting}
\end{Shaded}

\bibliography{book.bib,packages.bib}


\end{document}
